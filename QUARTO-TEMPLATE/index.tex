% Options for packages loaded elsewhere
% Options for packages loaded elsewhere
\PassOptionsToPackage{unicode}{hyperref}
\PassOptionsToPackage{hyphens}{url}
\PassOptionsToPackage{dvipsnames,svgnames,x11names}{xcolor}
%
\documentclass[
  letterpaper,
]{article}
\usepackage{xcolor}
\usepackage[top=2.5cm,bottom=2.5cm,left=2.5cm,right=2.5cm]{geometry}
\usepackage{amsmath,amssymb}
\setcounter{secnumdepth}{5}
\usepackage{iftex}
\ifPDFTeX
  \usepackage[T1]{fontenc}
  \usepackage[utf8]{inputenc}
  \usepackage{textcomp} % provide euro and other symbols
\else % if luatex or xetex
  \usepackage{unicode-math} % this also loads fontspec
  \defaultfontfeatures{Scale=MatchLowercase}
  \defaultfontfeatures[\rmfamily]{Ligatures=TeX,Scale=1}
\fi
\usepackage{lmodern}
\ifPDFTeX\else
  % xetex/luatex font selection
\fi
% Use upquote if available, for straight quotes in verbatim environments
\IfFileExists{upquote.sty}{\usepackage{upquote}}{}
\IfFileExists{microtype.sty}{% use microtype if available
  \usepackage[]{microtype}
  \UseMicrotypeSet[protrusion]{basicmath} % disable protrusion for tt fonts
}{}
\makeatletter
\@ifundefined{KOMAClassName}{% if non-KOMA class
  \IfFileExists{parskip.sty}{%
    \usepackage{parskip}
  }{% else
    \setlength{\parindent}{0pt}
    \setlength{\parskip}{6pt plus 2pt minus 1pt}}
}{% if KOMA class
  \KOMAoptions{parskip=half}}
\makeatother
% Make \paragraph and \subparagraph free-standing
\makeatletter
\ifx\paragraph\undefined\else
  \let\oldparagraph\paragraph
  \renewcommand{\paragraph}{
    \@ifstar
      \xxxParagraphStar
      \xxxParagraphNoStar
  }
  \newcommand{\xxxParagraphStar}[1]{\oldparagraph*{#1}\mbox{}}
  \newcommand{\xxxParagraphNoStar}[1]{\oldparagraph{#1}\mbox{}}
\fi
\ifx\subparagraph\undefined\else
  \let\oldsubparagraph\subparagraph
  \renewcommand{\subparagraph}{
    \@ifstar
      \xxxSubParagraphStar
      \xxxSubParagraphNoStar
  }
  \newcommand{\xxxSubParagraphStar}[1]{\oldsubparagraph*{#1}\mbox{}}
  \newcommand{\xxxSubParagraphNoStar}[1]{\oldsubparagraph{#1}\mbox{}}
\fi
\makeatother

\usepackage{color}
\usepackage{fancyvrb}
\newcommand{\VerbBar}{|}
\newcommand{\VERB}{\Verb[commandchars=\\\{\}]}
\DefineVerbatimEnvironment{Highlighting}{Verbatim}{commandchars=\\\{\}}
% Add ',fontsize=\small' for more characters per line
\usepackage{framed}
\definecolor{shadecolor}{RGB}{241,243,245}
\newenvironment{Shaded}{\begin{snugshade}}{\end{snugshade}}
\newcommand{\AlertTok}[1]{\textcolor[rgb]{0.68,0.00,0.00}{#1}}
\newcommand{\AnnotationTok}[1]{\textcolor[rgb]{0.37,0.37,0.37}{#1}}
\newcommand{\AttributeTok}[1]{\textcolor[rgb]{0.40,0.45,0.13}{#1}}
\newcommand{\BaseNTok}[1]{\textcolor[rgb]{0.68,0.00,0.00}{#1}}
\newcommand{\BuiltInTok}[1]{\textcolor[rgb]{0.00,0.23,0.31}{#1}}
\newcommand{\CharTok}[1]{\textcolor[rgb]{0.13,0.47,0.30}{#1}}
\newcommand{\CommentTok}[1]{\textcolor[rgb]{0.37,0.37,0.37}{#1}}
\newcommand{\CommentVarTok}[1]{\textcolor[rgb]{0.37,0.37,0.37}{\textit{#1}}}
\newcommand{\ConstantTok}[1]{\textcolor[rgb]{0.56,0.35,0.01}{#1}}
\newcommand{\ControlFlowTok}[1]{\textcolor[rgb]{0.00,0.23,0.31}{\textbf{#1}}}
\newcommand{\DataTypeTok}[1]{\textcolor[rgb]{0.68,0.00,0.00}{#1}}
\newcommand{\DecValTok}[1]{\textcolor[rgb]{0.68,0.00,0.00}{#1}}
\newcommand{\DocumentationTok}[1]{\textcolor[rgb]{0.37,0.37,0.37}{\textit{#1}}}
\newcommand{\ErrorTok}[1]{\textcolor[rgb]{0.68,0.00,0.00}{#1}}
\newcommand{\ExtensionTok}[1]{\textcolor[rgb]{0.00,0.23,0.31}{#1}}
\newcommand{\FloatTok}[1]{\textcolor[rgb]{0.68,0.00,0.00}{#1}}
\newcommand{\FunctionTok}[1]{\textcolor[rgb]{0.28,0.35,0.67}{#1}}
\newcommand{\ImportTok}[1]{\textcolor[rgb]{0.00,0.46,0.62}{#1}}
\newcommand{\InformationTok}[1]{\textcolor[rgb]{0.37,0.37,0.37}{#1}}
\newcommand{\KeywordTok}[1]{\textcolor[rgb]{0.00,0.23,0.31}{\textbf{#1}}}
\newcommand{\NormalTok}[1]{\textcolor[rgb]{0.00,0.23,0.31}{#1}}
\newcommand{\OperatorTok}[1]{\textcolor[rgb]{0.37,0.37,0.37}{#1}}
\newcommand{\OtherTok}[1]{\textcolor[rgb]{0.00,0.23,0.31}{#1}}
\newcommand{\PreprocessorTok}[1]{\textcolor[rgb]{0.68,0.00,0.00}{#1}}
\newcommand{\RegionMarkerTok}[1]{\textcolor[rgb]{0.00,0.23,0.31}{#1}}
\newcommand{\SpecialCharTok}[1]{\textcolor[rgb]{0.37,0.37,0.37}{#1}}
\newcommand{\SpecialStringTok}[1]{\textcolor[rgb]{0.13,0.47,0.30}{#1}}
\newcommand{\StringTok}[1]{\textcolor[rgb]{0.13,0.47,0.30}{#1}}
\newcommand{\VariableTok}[1]{\textcolor[rgb]{0.07,0.07,0.07}{#1}}
\newcommand{\VerbatimStringTok}[1]{\textcolor[rgb]{0.13,0.47,0.30}{#1}}
\newcommand{\WarningTok}[1]{\textcolor[rgb]{0.37,0.37,0.37}{\textit{#1}}}

\usepackage{longtable,booktabs,array}
\usepackage{calc} % for calculating minipage widths
% Correct order of tables after \paragraph or \subparagraph
\usepackage{etoolbox}
\makeatletter
\patchcmd\longtable{\par}{\if@noskipsec\mbox{}\fi\par}{}{}
\makeatother
% Allow footnotes in longtable head/foot
\IfFileExists{footnotehyper.sty}{\usepackage{footnotehyper}}{\usepackage{footnote}}
\makesavenoteenv{longtable}
\usepackage{graphicx}
\makeatletter
\newsavebox\pandoc@box
\newcommand*\pandocbounded[1]{% scales image to fit in text height/width
  \sbox\pandoc@box{#1}%
  \Gscale@div\@tempa{\textheight}{\dimexpr\ht\pandoc@box+\dp\pandoc@box\relax}%
  \Gscale@div\@tempb{\linewidth}{\wd\pandoc@box}%
  \ifdim\@tempb\p@<\@tempa\p@\let\@tempa\@tempb\fi% select the smaller of both
  \ifdim\@tempa\p@<\p@\scalebox{\@tempa}{\usebox\pandoc@box}%
  \else\usebox{\pandoc@box}%
  \fi%
}
% Set default figure placement to htbp
\def\fps@figure{htbp}
\makeatother


% definitions for citeproc citations
\NewDocumentCommand\citeproctext{}{}
\NewDocumentCommand\citeproc{mm}{%
  \begingroup\def\citeproctext{#2}\cite{#1}\endgroup}
\makeatletter
 % allow citations to break across lines
 \let\@cite@ofmt\@firstofone
 % avoid brackets around text for \cite:
 \def\@biblabel#1{}
 \def\@cite#1#2{{#1\if@tempswa , #2\fi}}
\makeatother
\newlength{\cslhangindent}
\setlength{\cslhangindent}{1.5em}
\newlength{\csllabelwidth}
\setlength{\csllabelwidth}{3em}
\newenvironment{CSLReferences}[2] % #1 hanging-indent, #2 entry-spacing
 {\begin{list}{}{%
  \setlength{\itemindent}{0pt}
  \setlength{\leftmargin}{0pt}
  \setlength{\parsep}{0pt}
  % turn on hanging indent if param 1 is 1
  \ifodd #1
   \setlength{\leftmargin}{\cslhangindent}
   \setlength{\itemindent}{-1\cslhangindent}
  \fi
  % set entry spacing
  \setlength{\itemsep}{#2\baselineskip}}}
 {\end{list}}
\usepackage{calc}
\newcommand{\CSLBlock}[1]{\hfill\break\parbox[t]{\linewidth}{\strut\ignorespaces#1\strut}}
\newcommand{\CSLLeftMargin}[1]{\parbox[t]{\csllabelwidth}{\strut#1\strut}}
\newcommand{\CSLRightInline}[1]{\parbox[t]{\linewidth - \csllabelwidth}{\strut#1\strut}}
\newcommand{\CSLIndent}[1]{\hspace{\cslhangindent}#1}



\setlength{\emergencystretch}{3em} % prevent overfull lines

\providecommand{\tightlist}{%
  \setlength{\itemsep}{0pt}\setlength{\parskip}{0pt}}



 


\makeatletter
\@ifpackageloaded{bookmark}{}{\usepackage{bookmark}}
\makeatother
\makeatletter
\@ifpackageloaded{caption}{}{\usepackage{caption}}
\AtBeginDocument{%
\ifdefined\contentsname
  \renewcommand*\contentsname{Table of contents}
\else
  \newcommand\contentsname{Table of contents}
\fi
\ifdefined\listfigurename
  \renewcommand*\listfigurename{List of Figures}
\else
  \newcommand\listfigurename{List of Figures}
\fi
\ifdefined\listtablename
  \renewcommand*\listtablename{List of Tables}
\else
  \newcommand\listtablename{List of Tables}
\fi
\ifdefined\figurename
  \renewcommand*\figurename{Figure}
\else
  \newcommand\figurename{Figure}
\fi
\ifdefined\tablename
  \renewcommand*\tablename{Table}
\else
  \newcommand\tablename{Table}
\fi
}
\@ifpackageloaded{float}{}{\usepackage{float}}
\floatstyle{ruled}
\@ifundefined{c@chapter}{\newfloat{codelisting}{h}{lop}}{\newfloat{codelisting}{h}{lop}[chapter]}
\floatname{codelisting}{Listing}
\newcommand*\listoflistings{\listof{codelisting}{List of Listings}}
\makeatother
\makeatletter
\makeatother
\makeatletter
\@ifpackageloaded{caption}{}{\usepackage{caption}}
\@ifpackageloaded{subcaption}{}{\usepackage{subcaption}}
\makeatother
\usepackage{bookmark}
\IfFileExists{xurl.sty}{\usepackage{xurl}}{} % add URL line breaks if available
\urlstyle{same}
\hypersetup{
  pdftitle={Avaliação da Funcionalidade do Solo},
  pdfauthor={Diego et al.},
  colorlinks=true,
  linkcolor={blue},
  filecolor={Maroon},
  citecolor={Blue},
  urlcolor={Blue},
  pdfcreator={LaTeX via pandoc}}


\title{Avaliação da Funcionalidade do Solo}
\usepackage{etoolbox}
\makeatletter
\providecommand{\subtitle}[1]{% add subtitle to \maketitle
  \apptocmd{\@title}{\par {\large #1 \par}}{}{}
}
\makeatother
\subtitle{Sistemas de Manejo Agrícola Usando Lógica Fuzzy}
\author{Diego et al.}
\date{2025-10-31}
\begin{document}
\maketitle

\renewcommand*\contentsname{Table of contents}
{
\hypersetup{linkcolor=}
\setcounter{tocdepth}{2}
\tableofcontents
}

\bookmarksetup{startatroot}

\chapter*{Resumo}\label{resumo}

\markboth{Resumo}{Resumo}

A conversão de ecossistemas naturais em sistemas agrícolas altera de
forma substantiva a dinâmica do fósforo (P) e do nitrogênio (N) no solo,
com implicações diretas para a sustentabilidade de ambientes tropicais.
Este estudo avaliou a dinâmica funcional de N e P em frações lábeis e
húmicas sob quatro usos da terra no sul do Tocantins (Cerrado nativo,
Eucalipto, Pastagem e Agricultura com milho), visando integrar
evidências preditivas, estruturais e sínteses fuzzy para diagnóstico da
funcionalidade edáfica. Amostras de 0--50 cm foram analisadas quanto às
frações lábeis (NLabil, NMOL; PLabil, PMOL), húmicas (NTAF, NTAH, NTHum;
PTAF, PTAH, PTHum) e totais (NT, PT), incluindo estoques (EstNT, EstPT).
A regressão por mínimos quadrados parciais (PLSR) identificou as frações
mais relevantes por escores VIP, a modelagem por equações estruturais
(PLS-SEM) explicitou relações causais entre construtos latentes
(N\_lábil, N\_húmico, P\_lábil, P\_húmico) e os estoques totais
(N\_total, P\_total), e um sistema de inferência fuzzy (FSNSI)
sintetizou N\_total, P\_total e densidade do solo em um índice único de
funcionalidade.

As frações húmicas emergiram como preditoras dominantes (VIP
\textgreater{} 1,0) e apresentaram efeito positivo forte sobre os
estoques totais (β = 1,286), ao passo que as frações lábeis exibiram
coeficientes negativos moderados (β = −0,313). O PLS-SEM indicou ajuste
elevado (R² \textgreater{} 0,80; SRMR \textless{} 0,06) e simetria dos
coeficientes entre N e P, sugerindo acoplamento biogeoquímico entre seus
ciclos. O FSNSI diferenciou os usos: o Eucalipto apresentou a maior
funcionalidade (FSNSI = 6,07), superando o Cerrado (4,42; p = 0,036),
enquanto Agricultura e Pastagem revelaram funcionalidade intermediária.
A densidade do solo correlacionou-se negativamente com o FSNSI (r =
−0,424; p \textless{} 0,001), confirmando-se como fator limitante
transversal. Em conjunto, os resultados indicam que o enriquecimento das
frações húmicas, associado a maior aporte e qualidade da serapilheira e
a melhor estrutura física, sustenta os estoques de N e P e a
funcionalidade edáfica. Conclui-se que a integração entre PLSR, PLS-SEM
e inferência fuzzy constitui abordagem robusta e multiescalar para
diagnosticar a sustentabilidade nutricional do solo e orientar
intervenções conservacionistas em ambientes tropicais.

\textbf{Palavras-chave:} Nitrogênio do solo; Frações húmicas; Mudança de
uso da terra; Cerrado; Latossolo; PLS-SEM; PLSR.

\section*{Estrutura do Documento}\label{estrutura-do-documento}
\addcontentsline{toc}{section}{Estrutura do Documento}

\markright{Estrutura do Documento}

Este trabalho segue a
\href{https://www.projecttier.org/tier-protocol/protocol-4-0/}{TIER
Protocol 4.0} para garantir clareza e reprodutibilidade:

\begin{itemize}
\tightlist
\item
  \textbf{Dados} (\texttt{data/}): Conjunto de dados brutos e
  processados
\item
  \textbf{Código} (\texttt{code/}): Scripts R para análise estatística
\item
  \textbf{Outputs} (\texttt{outputs/}): Resultados e tabelas geradas
\item
  \textbf{Figuras} (\texttt{figures/}): Gráficos e visualizações
\item
  \textbf{Capítulos} (\texttt{chapters/}): Seções do artigo
\end{itemize}

\section*{Como Reproduzir}\label{como-reproduzir}
\addcontentsline{toc}{section}{Como Reproduzir}

\markright{Como Reproduzir}

Para reproduzir todos os resultados:

\begin{Shaded}
\begin{Highlighting}[]
\ExtensionTok{quarto}\NormalTok{ render}
\end{Highlighting}
\end{Shaded}

Para gerar apenas HTML:

\begin{Shaded}
\begin{Highlighting}[]
\ExtensionTok{quarto}\NormalTok{ render }\AttributeTok{{-}{-}to}\NormalTok{ html}
\end{Highlighting}
\end{Shaded}

Para gerar Word:

\begin{Shaded}
\begin{Highlighting}[]
\ExtensionTok{quarto}\NormalTok{ render }\AttributeTok{{-}{-}to}\NormalTok{ docx}
\end{Highlighting}
\end{Shaded}

\section*{Requisitos}\label{requisitos}
\addcontentsline{toc}{section}{Requisitos}

\markright{Requisitos}

\begin{itemize}
\tightlist
\item
  \href{https://quarto.org/}{Quarto} ≥ 1.4
\item
  \href{https://www.r-project.org/}{R} ≥ 4.0
\item
  Pandoc (incluído no Quarto)
\end{itemize}

\section*{Citação}\label{citauxe7uxe3o}
\addcontentsline{toc}{section}{Citação}

\markright{Citação}

Cite este trabalho como:

\begin{Shaded}
\begin{Highlighting}[]
\VariableTok{@article}\NormalTok{\{}\OtherTok{vidal2025}\NormalTok{,}
  \DataTypeTok{title}\NormalTok{ = \{Modelagem Fuzzy{-}PLS da Dinâmica de Fósforo e Nitrogênio em Usos do Solo no Sul do Tocantins\},}
  \DataTypeTok{author}\NormalTok{ = \{Vidal, Diego\},}
  \DataTypeTok{year}\NormalTok{ = \{2025\}}
\NormalTok{\}}
\end{Highlighting}
\end{Shaded}

\begin{center}\rule{0.5\linewidth}{0.5pt}\end{center}

\textbf{Última atualização}: \texttt{r\ Sys.Date()}

\bookmarksetup{startatroot}

\chapter{Introdução}\label{introduuxe7uxe3o}

Os solos tropicais constituem sistemas biogeoquímicos dinâmicos, onde a
matéria orgânica regula os ciclos de nutrientes e sustenta a
estabilidade estrutural (Cotrufo et al., 2019, Lal, 2020). A conversão
de ecossistemas nativos para usos agropecuários e silviculturais, no
entanto, altera o aporte e a qualidade dos resíduos orgânicos,
acelerando a decomposição e reduzindo a formação de compostos húmicos
estáveis. Isso resulta na diminuição dos estoques de nitrogênio (N) e
fósforo (P), comprometendo a resiliência do solo e a ciclagem de
nutrientes em longo prazo (Tivet et al., 2013, Silva et al., 2022, Wang
et al., 2023). Tais mudanças são particularmente evidentes no bioma
Cerrado, onde a intensificação do uso da terra tem levado à degradação
da funcionalidade edáfica (Strassburg et al., 2017, Sousa et al., 2021).

O Cerrado brasileiro, reconhecido como um dos principais hotspots de
biodiversidade do planeta, abriga solos altamente intemperizados,
naturalmente ácidos e pobres em fósforo e nitrogênio disponível (Sano et
al., 2020). Nessas condições, a sustentabilidade do sistema depende
fortemente da manutenção das frações húmicas da MOS, que atuam como
reservatórios de nutrientes e contribuem para a formação de agregados
estáveis e a retenção de água (Lehmann \& Kleber, 2015, Paul, 2016). As
substâncias húmicas compostas por ácidos húmicos (AH), ácidos fúlvicos
(AF) e humina (Hum) exercem papel essencial na estabilização de N e P,
seja por adsorção, complexação ou imobilização biogênica, promovendo a
persistência desses elementos no solo (Stevenson, 1994, Helfenstein et
al., 2020).

Nos últimos anos, tem-se intensificado o interesse científico em
compreender o acoplamento biogeoquímico entre N e P isto é, como suas
formas lábeis e húmicas interagem e se co-estabilizam na MOS. Evidências
crescentes indicam que esses nutrientes não atuam isoladamente, mas são
co-regulados por processos de decomposição, humificação e proteção
físico-química em complexos organo-minerais (Chen et al., 2018,
Helfenstein et al., 2020). Esse acoplamento manifesta-se através da
incorporação simultânea de N e P na biomassa microbiana durante a
decomposição de resíduos vegetais, com subsequente liberação
sincronizada mediante lise celular. Simultaneamente, ocorre adsorção
competitiva de formas inorgânicas (NH₄⁺, NO₃⁻, H₂PO₄⁻) e orgânicas
(aminoácidos, nucleotídeos, fosfolipídios) em superfícies minerais e
colóides orgânicos, onde a presença de um nutriente modula a
disponibilidade do outro através de competição por sítios de ligação.
Adicionalmente, a formação de complexos ternários envolvendo substâncias
húmicas, cátions polivalentes (Ca²⁺, Fe³⁺, Al³⁺) e ânions nutricionais
estabiliza simultaneamente N e P em formas de longa persistência,
enquanto a co-oclusão física em microagregados estabilizados por humina
protege N e P orgânicos contra mineralização rápida. Esse acoplamento
implica que perturbações no ciclo de um nutriente, criando desbalanços
estequiométricos que podem comprometer a eficiência de uso de nutrientes
e a estabilidade da MOS. Apesar disso, a maioria dos estudos ainda
analisa N e P de forma independente, negligenciando as relações
estruturais entre seus compartimentos e as implicações para a
sustentabilidade edáfica (Marinho Junior et al., 2021, Wang et al.,
2023).

Modelos analíticos avançados, como a modelagem por equações estruturais
baseada em mínimos quadrados parciais (PLS-SEM), oferecem novas
oportunidades para desvendar as relações causais entre compartimentos
edáficos (Hair et al., 2021). Em paralelo, métodos de inteligência
computacional, como os sistemas de inferência fuzzy, possibilitam
sintetizar múltiplos indicadores químicos e físicos do solo em um único
índice de desempenho funcional (Mamdani, 1977, Lima et al., 2023). A
combinação dessas abordagens fornece uma visão abrangente da
funcionalidade edáfica, considerando simultaneamente a disponibilidade
química de nutrientes e as restrições físicas impostas pelo uso e manejo
da terra, permitindo traduzir os resultados em um Índice Fuzzy de
Sustentabilidade Edáfica (FESI) (Mendonça et al., 2024).

A compreensão integrada da dinâmica funcional de N e P em diferentes
usos da terra é, portanto, fundamental para orientar estratégias de
manejo conservacionista e restauração ecológica em ambientes tropicais.
A estabilidade da MOS, mais do que a simples quantidade de matéria
orgânica acumulada, depende da proporção entre frações lábeis de rápida
renovação e húmicas de longa persistência, as quais determinam o
equilíbrio entre disponibilidade imediata e armazenamento de longo prazo
de nutrientes (Cotrufo et al., 2019). Assim, avaliar como as pressões
antrópicas alteram essa relação é essencial para estimar a capacidade do
solo de sustentar funções ecológicas críticas e resistir à degradação.

Com base nesse contexto, formula-se a hipótese de que as frações húmicas
e lábeis de nitrogênio e fósforo exercem contribuições distintas, porém
complementares, para os estoques totais desses nutrientes no solo, e que
a degradação física, expressa pela densidade aparente, atua como fator
limitante da funcionalidade edáfica. Espera-se que sistemas sob
vegetação nativa apresentem maior co-estabilização de N e P em frações
húmicas estáveis, enquanto usos agrícolas e silviculturais revelem
predominância de formas lábeis e menor sinergia entre os ciclos
biogeoquímicos desses elementos.

Diante dessas premissas, o presente estudo teve como objetivo avaliar a
dinâmica funcional de nitrogênio e fósforo nas frações húmicas e lábeis
do solo sob diferentes usos da terra no sul do Tocantins, integrando
abordagens de modelagem preditiva, estrutural e de inferência fuzzy para
diagnóstico da sustentabilidade edáfica. Para isso, buscou-se
identificar as frações de N e P mais relevantes para predição dos
estoques totais por meio de regressão por mínimos quadrados parciais
(PLSR), quantificar as contribuições relativas das formas lábeis e
húmicas aos estoques totais mediante modelagem por equações estruturais
(PLS-SEM) e análise multigrupo entre usos da terra e integrar atributos
químicos e físicos do solo em um Índice Fuzzy de Sustentabilidade
Edáfica (FSNSI) para sintetizar a funcionalidade biogeoquímica dos
sistemas avaliados.

\bookmarksetup{startatroot}

\chapter{Metodologia}\label{metodologia}

\section{Área de estudo}\label{uxe1rea-de-estudo}

A pesquisa foi desenvolvida, no município de São Valério da Natividade
em Tocantins, com área total de 53,23 ha, nas coordenadas geográficas
11º54'37'' S e 48º12'31'' O (Figura 1).

\begin{figure}

{\centering \includegraphics[width=0.7\linewidth,height=\textheight,keepaspectratio]{chapters/../../../2-FIGURAS/mapa_area.png}

}

\caption{Figura 1. Localização das áreas experimentais}

\end{figure}%

O clima do município, é do tipo úmido subsumido com moderada deficiência
hídrica (C2wA'a''), com duas estações bem definidas, inverno seco que
vai de maio a outubro e verão chuvoso, de novembro a abril
(Thornthwaite, 1948). De acordo com a classificação climática de Köppen,
o município de São Valério da Natividade possui pluviosidade média anual
de 1643.3 mm, temperatura média 26°C. Está localizado a uma altitude de
320.48 m, e possui 90\% da região com declividade inferior a 8º,
indicando solos planos (Cho et al., 2021). Os solos das áreas
experimentais foram classificados como Latossolo Vermelho-Amarelo,
possui uma estrutura granular a subagregada, com boa a excelente
drenagem, porém é naturalmente pobre em nutrientes, especialmente
nitrogênio, fósforo e potássio (Lopes \& Guilherme, 1996, Ker, 1997).

\section{Características das áreas de
estudo}\label{caracteruxedsticas-das-uxe1reas-de-estudo}

A área preservada com Cerrado (Sensu Stricto) (Figura 2a) considerada
como testemunha possui 44,82 ha, com mais 40 anos, localizada nas
coordenadas 11°54'57'\,`S e 48°11'59'\,'W (Figura 2). A vegetação possui
características do tipo denso, com árvore que variam entre 5 a 8 metros
de altura, uma vez que, a vegetação do cerrado sensu stricto não possui
uma fisionomia única, pelo contrário, é bastante diversificada,
apresentando desde formas campestres bem abertas, até formas
relativamente densas (florestais) (Klink \& Machado, 2005, Sano et al.,
2019).

Figura 2. Áreas estudadas. Fotomontagem das áreas: (a) Cerrado Stricto
Sensu (vegetação nativa preservada); (b) Eucalipto (\emph{Eucalyptus}
sp.); (c) Mogno Africano (\emph{Khaya ivorensis}); (d) Teca
(\emph{Tectona grandis}); (e) Agricultura (soja/milho em rotação).

\begin{longtable}[]{@{}
  >{\centering\arraybackslash}p{(\linewidth - 4\tabcolsep) * \real{0.3333}}
  >{\centering\arraybackslash}p{(\linewidth - 4\tabcolsep) * \real{0.3333}}
  >{\centering\arraybackslash}p{(\linewidth - 4\tabcolsep) * \real{0.3333}}@{}}
\toprule\noalign{}
\begin{minipage}[b]{\linewidth}\centering
(a)
\end{minipage} & \begin{minipage}[b]{\linewidth}\centering
(b)
\end{minipage} & \begin{minipage}[b]{\linewidth}\centering
(c)
\end{minipage} \\
\midrule\noalign{}
\endhead
\bottomrule\noalign{}
\endlastfoot
\includegraphics[width=0.35\linewidth,height=\textheight,keepaspectratio]{chapters/../../../2-FIGURAS/cerrado.jpg}
&
\includegraphics[width=0.35\linewidth,height=\textheight,keepaspectratio]{chapters/../../../2-FIGURAS/eucalipto.jpg}
&
\includegraphics[width=0.35\linewidth,height=\textheight,keepaspectratio]{chapters/../../../2-FIGURAS/mogno.jpg} \\
(d) & (e) & \\
\includegraphics[width=0.35\linewidth,height=\textheight,keepaspectratio]{chapters/../../../2-FIGURAS/teca.jpg}
&
\includegraphics[width=0.35\linewidth,height=\textheight,keepaspectratio]{chapters/../../../2-FIGURAS/agricultura.jpg}
& \\
\end{longtable}

Um inventario foi realizado e calculado os parâmetros fitossociológicos:
Densidade Relativa - DR, Dominância Relativa - DMR, Frequência Relativa
-- FR e Índice de Valor de Importância - IVI (Queiroz et al., 2017)
(Apêndice A).

O plantio de eucalipto (\emph{Eucalyptus} sp.) possui área total de 2,29
ha com cinco anos de idade, localizado nas coordenadas de 11°54'32'\,'S,
48°12'22'\,'W (Figura 2b). Em relação a densidade do plantio, foi de
aproximadamente 1667 mudas/ha em um espaçamento de 3 x 2 m, em covas que
possuem dimensões de 0,40 x 0,40 x 0,40 m, abertas manualmente com
auxílio de cavadeiras, seguido de aplicação de adubo NPK, na formulação
5-25-15, com intuito de estimular o desenvolvimento vegetal, para melhor
acondicionamento das mudas ao solo (Verai et al., 2022). No início do
plantio foi realizado adubação de base com 20 kg de Ca2+, 0,8 kg de
Zn2+, 12 kg de S (SO42-), 1,6 kg de Cu2+ e 1,6 kg de B (H3BO3). A
limpeza da área, para remoção de vegetação nativa, foi realizada com o
auxílio de lâmina frontal acoplada ao trator de esteira, sucedido por
aragem e gradagem, permitindo maior interação de oxigênio ao solo, o que
viabiliza melhorias nos processos químicos e biológicos (Almeida et al.,
2024). No que tange ao processo de alinhamento e esquadrejamento das
mudas, optou-se pela utilização do método do Triangulo Reto (3/4/5) com
auxílio de baliza e trena (Verai et al., 2022). Já o plantio de Mogno
Africano (\emph{Khaya ivorensis}) possui uma área total de 1,94 ha, com
sete anos de idade, localizado nas coordenadas 11°54'29'\,`S,
48°12'10'`W (Figura 2c). O plantio foi realizado em covas com dimensões
de 0,40 x 0,40 x 0,40 m, e espaçamentos de 3 x 3 m com densidade de 1111
mudas/ha, e durante o plantio foi adicionado em cada cova 0,2 kg de
adubo NPK com formulação 00-10-10, e 5 kg de esterco bovino curtido,
sendo aplicados durante o primeiro ano de vida a cada três meses, além
disto foi realizado o desbaste, quando as copas começaram a se
encontrem, deixando o espaçamento final de 6 x 6 metros (Silva \&
Barreira, 2023). Assim como na área de Eucalipto a limpeza da área, foi
realizada com o auxílio de lâmina frontal acoplada ao trator de esteira,
sucedido por serviços de aragem e gradagem (Campos \& Montanari, 2024).
O Plantio de Teca (\emph{Tectona grandis}) possui uma área total de 1,12
ha, com dez anos de idade, localizado nas coordenadas 11°54'22'\,'S e
48°12'17'\,'W (Figura 2d). A limpeza da área antes do plantio, foi
realizada com o auxílio de lâmina frontal acoplada ao trator de esteira,
sucedido por serviços de aragem e gradagem (Campos \& Montanari, 2024).
Durante o primeiro ano, foram realizadas duas adubações de cobertura,
onde a primeira foi executada aos 60 dias e a outra no oitavo mês, com
aplicação de 95 g cova-1 de NPK com formulação de 20-05-20.

O plantio foi realizado manualmente em espaçamento de 3 x 2 m, nas
dimensões de 0,40 x 0,40 m, com 1.667 mudas/hectare, a qual, foram
instaladas individualmente com seu colo ao nível do solo. Foi realizado
também o replantio das mudas que falharam, além da limpeza no primeiro
ano, uma vez que, a teca é particularmente sensível à competição de
gramíneas, sendo assim, necessário manter o terreno bem carpido; no
segundo ano, o sombreamento proporcionado pela teca evitou boa parte do
desenvolvimento de ervas daninhas, reduzindo a necessidade de capinas e
roçadas e no terceiro ano, não foram mais necessários tratos culturais
(Moreira et al., 2021).

Além disso, foram realizadas atividades de podas e desbrotas com auxílio
de serrotes, com observações constantes quanto à dimensão dos galhos.
Por fim a área de agricultura possui 3,06 ha, possui mais de 10 anos,
localizada nas coordenadas de 11°54'44'\,'S e 48°12'02'\,'W (Figura 2e).
A área destinada a agricultura possui mais de 10 anos, sendo rotacionado
entre o cultivo de milho e soja. No momento da coleta, a área estava
sendo utilizada para plantio de soja que se estende entre os meses de
junho e setembro. Foram empregadas atividades de gradagem e nivelamento
do solo, com posterior aplicação de 300 kg ha-1 de NPK na formulação
4-28-10, com espaçamento entre os indivíduos de 0,5 m, tratados durante
todo o plantio com fungicidas tiofanto-metílico e azoxistrobina, e
inseticidas, na dosagem de 100 g para cada 50 kg de sementes, além da
remoção de ervas daninhas (Machado et al., 2024). Já o cultivo do milho,
estende-se entre os meses janeiro a março, em virtude das condições
favoráveis promovidas pela precipitação. O preparo do solo ocorreu
através de gradagem e sulcamento, com espaçamento médio, entre os
indivíduos, de 0,2 x 0,8 m e adubação de 400 kg de 4-14-18 por ha-1, no
interior do sulco de semeadura, e 50, 100 e 150 kg de N por ha-1, sobre
a superfície do solo, aliado aos procedimentos relacionados ao combate
de daninhas, por meio de capina manual e emprego de herbicidas, quando
necessário (Eckardt et al., 2021).

\section{Amostragem de solo}\label{amostragem-de-solo}

As cinco trincheiras em cada área estudada com dimensões com ajuda de um
gabarito 70 × 70 × 100 cm (Figura 3) em pontos distintos (Marinho Junior
et al., 2021), foram abertas manualmente, totalizando no total vinte e
cinco trincheiras.

\begin{figure}[H]

{\centering \pandocbounded{\includegraphics[keepaspectratio]{chapters/../../../2-FIGURAS/trincheira.png}}

}

\caption{Figura 3. Coleta de solo em área de vegetação nativa (Cerrado
sensu stricto)}

\end{figure}%

A coleta das amostras de solo, deformadas e indeformadas, para as
analises físicas e químicas foram realizadas nas profundidades de 0-10,
10-20, 20-30, 30-40, 40-50, 50-60, 60-80, 80-100 cm. Depois, as amostras
de solos deformadas foram secas ao ar e passada em peneira de 2 mm para
posterior analises. \#\#\# 2.6.1 Análises físicas

A determinação da distribuição dos tamanhos das partículas do solo foi
realizada em amostras deformadas por meio do método da pipeta (Teixeira
et al., 2017) (TABELA 2). A determinação da densidade do solo foi
realizada por meio do método do cilindro volumétrico, conforme descrito
por Teixeira et al. (2017) (APENDICE B). Desta forma, cilindros de aço
inoxidável com 3 cm de diâmetro e 8 cm de altura foram inseridos e
retirados do solo sem ocorrer deformação e acondicionadas em sacos
plásticos devidamente identificados. As amostras indeformadas foram
submetidas à secagem por 72 horas, utilizando-se uma estufa de
ventilação forçada com temperatura regulada para 105ºC, sendo
determinada a massa de solo seco para realização do cálculo de
densidade.

\subsection{2.6.2 Análises químicas}\label{anuxe1lises-quuxedmicas}

\subsubsection{2.6.2.1 Determinação do fósforo total no
solo}\label{determinauxe7uxe3o-do-fuxf3sforo-total-no-solo}

O solo foi pesado em 0,5 g, passado em peneira de 150 μm (100 mesh),
depois foi realizado a adição do ácido sulfúrico mais o peróxido de
hidrogênio, e colocado em bloco digestor. Através de colorimetria pelo
método de Murphy e Riley (1962) foi determinado P total do solo.

\subsubsection{2.6.2.2 Fracionamento das substâncias húmicas e teores de
fósforo em cada
fração}\label{fracionamento-das-substuxe2ncias-huxfamicas-e-teores-de-fuxf3sforo-em-cada-frauxe7uxe3o}

Para extrair as substâncias húmicas, as amostras de solo passaram por um
processo de fracionamento seguindo o método da International Humic
Substances Society (IHSS), conforme descrito por SWIFT em 1996. Esse
processo resultou na obtenção das frações de ácidos fúlvicos (AF),
ácidos húmicos (AH) e humina (Hum), baseado na solubilidade em soluções
ácidas e alcalinas. A extração começou com a mistura de 200 gramas de
solo com uma solução de HCl a 0,1 mol/L, na proporção de 1 grama de solo
para 10 mililitros de solução. Essa mistura foi agitada manualmente por
1 hora e, em seguida, deixada em repouso por 4 horas. O líquido que
ficou por cima foi sifonado e reservado, dando origem ao extrato I de
ácidos fúlvicos. Depois disso, adicionou-se uma solução de NaOH a 0,1
mol/L na mesma proporção (1:10) e também foi agitada manualmente. Após
essa etapa, a solução foi deixada em repouso por 16 horas. Em seguida, o
material que se precipitou foi separado, correspondendo à fração de
humina e ao material mineral. O sobrenadante, que continha as frações de
AF e AH, foi centrifugado por 10 minutos a 10.000 rpm. Depois,
acidificou-se essa solução com a adição de 50 mililitros de HCl a 6
mol/L, ajustando o pH para um valor entre 1 e 2, e agitou-se manualmente
por dois minutos. Após esse procedimento, a solução foi deixada em
repouso por 12 horas. Ao final, o sobrenadante foi desviado, obtendo-se
o extrato II de ácidos fúlvicos, enquanto o material precipitado estava
relacionado à fração de ácidos húmicos. Após o fracionamento das
substâncias húmicas, as amostras foram congeladas e liofilizadas para a
determinação dos teores de P em cada fração AF, AH e Hum, utilizando o
método de colorimetria pelo método de Murphy e Riley (1962).

\subsubsection{2.6.2.3 Determinação dos estoques de
fósforo}\label{determinauxe7uxe3o-dos-estoques-de-fuxf3sforo}

A partir das concentrações de P obtidas no solo e em cada uma das
frações húmicas, foi possível determinar os estoques no solo e frações
húmicas, expresso em microgramas por hectare (Mg ha⁻¹), para cada
profundidade amostrada, conforme equação a seguir: EstP = TP x Ds x e Em
que: EstP = Estoque de P na camada do solo, em Mg ha-1; TP = Teor de P
na fração amostrada de solo, em g kg-1; Ds = Densidade do solo, em g
cm-3; e = espessura da camada, em cm.

Após o cálculo do estoque de P em cada camada, foi realizada a correção
do estoque de P. Por fim, o estoque total de P no solo e nas substâncias
húmicas na profundidade de 0 a 50 cm foi resultante da soma dos valores
obtidos em cada camada amostrada.

\subsection{2.6.3 Determinação do Nitrogênio total no
solo}\label{determinauxe7uxe3o-do-nitroguxeanio-total-no-solo}

As amostras de solo deformadas, depois de passadas em peneiras de 2 mm,
foram maceradas em almofariz de porcelana e pistilo até formar um pó
fino, e passadas em peneira de malha de 150 μm. Os teores de Nitrogênio
total - NT total no solo foram determinados através deste material, pelo
método de combustão a seco, utilizando analisador elementar (Modelo
PE-2400 Série II Perkin Elmer).

\subsection{2.6.4 Fracionamento das substâncias húmicas e teores de
Nitrogênio em cada
fração}\label{fracionamento-das-substuxe2ncias-huxfamicas-e-teores-de-nitroguxeanio-em-cada-frauxe7uxe3o}

Para extração das substâncias húmicas, as amostras de solo foram
submetidas ao fracionamento segundo o método da International Humic
Substances Society - IHSS (Swift, 1996), obtendo-se as frações
correspondentes aos ácidos fúlvicos - AF, ácidos húmicos - AH e humina -
HUM, com base na solubilidade diferencial em soluções alcalinas e
ácidas. Para a extração foi realizada com uma mistura de 200 g de solo
com solução de HCl 0,1 mol L-1 na proporção de 1 g de solo para cada 10
mL de solução, sendo agitada manualmente por 1 hora, ficando depois em
repouso por 4 horas. O extrato sobrenadante foi sifonado e reservado,
correspondendo ao extrato I de AF. Assim, a solução de NaOH 0,1 mol L-1
foi adicionada e precipitada na mesma proporção citada anteriormente
(1:10) e também realizada agitação manual. Após este período, a solução
foi deixada em repouso por 16 horas, seguindo na sequência para a nova
retirada da mistura, na qual o material precipitado foi separado,
correspondendo à fração Hum e fração mineral. O material sobrenadante,
referente às frações AF e AH, foi centrifugado por 10 minutos a 10.000
rpm, sendo depois acidificado pela adição de 50 mL de HCl 6 mol L-1 até
atingir o valor de pH entre 1 e 2 e agitado manualmente por dois
minutos. Após este procedimento, a solução foi deixada em repouso por 12
horas. Por fim, após separação por desvio do sobrenadante, referente ao
extrato II de AF, obteve-se o material precipitado que está relacionada
à fração de AH. Posteriormente ao fracionamento das substâncias húmicas,
as amostras foram congeladas e liofilizadas para determinação dos teores
de nitrogênio em cada fração húmica (AF, AH e Hum), a partir do método
de combustão a seco, utilizando um analisador elementar (Modelo PE-2400
Série II Perkin Elmer).

2.3.5 Determinação dos estoques de Nitrogênio Após a obter os teores de
N pelos métodos citados anteriormente, foi realizada a determinação do
estoque N no solo e nas frações húmicas em Mg ha-1, em cada profundidade
amostrada, conforme a equação (Equação 2) a seguir:

Est(N)= TN\emph{Ds}e

Em que: Est(N) = Estoque de N na camada do solo, em Mg ha-1; TN = Teor
de N na fração amostrada de solo, em g kg-1; Ds = Densidade do solo, em
g cm-3; e = espessura da camada, em cm. Após o cálculo do estoque de N
em cada camada, foi realizada a correção do estoque de N, levando em
consideração as diferenças na massa do solo (Sisti et al., 2004). Sendo
assim, o estoque total de N no solo e nas substâncias húmicas na
profundidade de 0 a 50 cm, sendo a última considerada a camada de
impedimento foi resultante da soma dos valores obtidos em cada camada
amostrada.

\subsection{2.6.5 Determinação dos estoques de
Nitrogênio}\label{determinauxe7uxe3o-dos-estoques-de-nitroguxeanio}

Após a obter os teores de N pelos métodos citados anteriormente, foi
realizada a determinação do estoque N no solo e nas frações húmicas em
Mg ha-1, em cada profundidade amostrada, conforme a equação a seguir:

Est(N)= TN\emph{Ds}e Em que: Est(N) = Estoque de N na camada do solo, em
Mg ha-1; TN = Teor de N na fração amostrada de solo, em g kg-1; Ds =
Densidade do solo, em g cm-3; e = espessura da camada, em cm.

Após o cálculo do estoque de N em cada camada, foi realizada a correção
do estoque de N, levando em consideração as diferenças na massa do solo
(Sisti et al., 2004). Sendo assim, o estoque total de N no solo e nas
substâncias húmicas na profundidade de 0 a 50 cm, sendo a última
considerada a camada de impedimento foi resultante da soma dos valores
obtidos em cada camada amostrada.

\section{Análises estatísticas}\label{anuxe1lises-estatuxedsticas}

Os parâmetros avaliados passaram pelos testes de normalidade de Shapiro
Wilk, em seguida, foram submetidos a uma análise de variância para
avaliar as diferenças entre os usos da terra e profundidades. A
comparação das médias foi realizada pelo teste de Tukey a 5\% de
significância e utilizado o software estatístico SISVAR (Ferreira,
2011).

\subsection{2.7.1 Regressão por Mínimos Quadrados Parciais
(PLSR)}\label{regressuxe3o-por-muxednimos-quadrados-parciais-plsr}

Para identificar as variáveis preditoras mais relevantes e compreender a
relação entre as frações de nitrogênio e fósforo com os estoques totais
desses nutrientes, foi empregada a técnica de Regressão por Mínimos
Quadrados Parciais (Partial Least Squares Regression - PLSR). Esta
abordagem multivariada é particularmente adequada para conjuntos de
dados com multicolinearidade entre variáveis explicativas, permitindo a
redução da dimensionalidade e a construção de componentes latentes
(variáveis latentes, LV) que maximizam a covariância entre os preditores
(X) e a variável resposta (Y).

Dois modelos PLSR independentes foram ajustados: um primeiro modelo para
predição do nitrogênio total (NT), utilizando como preditoras as
variáveis NLabil, NMOL, NTAF, NTAH, NTHum, EstNT, EstNLabil, EstNMOL,
EstNAF, EstNAH e EstNTHum; e um segundo modelo para predição do fósforo
total (PT), com as preditoras PLabil, PMOL, PTAF, PTAH, PTHum, EstPT,
EstPLabil, EstPMOL, EstPAF, EstPAH e EstPTHum. Ambos os modelos foram
estimados com duas componentes latentes (LV1 e LV2), utilizando
validação cruzada leave-one-out (LOO) para avaliação da capacidade
preditiva.

A análise foi conduzida no ambiente R (R Core Team, 2023), utilizando o
pacote \texttt{pls} (Mevik \& Wehrens, 2007). A variância explicada
acumulada foi calculada tanto para a matriz de preditores (R²X) quanto
para a variável resposta (R²Y). Os loadings das variáveis foram
normalizados e representados em biplots de correlação, permitindo
visualizar a estrutura de covariância entre as frações de N e P e os
estoques totais.

Adicionalmente, foram calculados os escores de importância das variáveis
na projeção (Variable Importance in Projection - VIP), conforme o
critério de Wold, para identificar quais frações contribuem mais
significativamente para a predição dos estoques totais. Variáveis com
VIP \textgreater{} 1.0 foram consideradas altamente relevantes, enquanto
VIP \textless{} 0.8 indicaram contribuição marginal ao modelo.

\section{Modelagem Proposta}\label{modelagem-proposta}

O presente estudo teve como objetivo avaliar o funcionamento
biogeoquímico do solo em sistemas contrastantes de uso da terra, com
ênfase na dinâmica funcional dos elementos nitrogênio (N) e fósforo (P),
integrando suas formas lábeis e húmicas aos estoques totais. Para isso,
foram adotadas duas abordagens complementares: a modelagem por equações
estruturais com mínimos quadrados parciais (PLS-SEM) e um sistema de
inferência fuzzy do tipo Mamdani, destinado à construção de um Índice
Fuzzy de Sustentação Nutricional Fuzzy (FSNSI). Nesse modelo, a
densidade do solo (Ds) foi incorporada como variável penalizadora,
assumindo que valores mais elevados de densidade refletem condições
físicas menos favoráveis à conservação da estrutura do solo e à
atividade biológica. Para tal, os dados de Ds foram normalizados de
forma invertida, de modo que maiores densidades recebessem notas mais
baixas na escala fuzzy, amplificando o impacto negativo dessa condição
sobre o valor final do índice.

Adicionalmente, as frações granulométricas do solo (areia, silte e
argila) foram introduzidas no processo analítico como covariáveis de
controle, com o objetivo de mitigar o viés potencial decorrente de
diferenças texturais naturais entre os ambientes avaliados. Embora essas
variáveis não tenham sido incorporadas diretamente ao sistema fuzzy como
entradas decisórias, sua normalização e análise paralela permitiram
contextualizar as respostas funcionais observadas, assegurando que os
efeitos inferidos para Ds, N total e P total refletissem
predominantemente condições associadas ao uso e manejo do solo, e não à
sua classe textural de origem.

O índice FSNSI foi estruturado com base em um sistema fuzzy do tipo
Mamdani, composto por três componentes formais principais: a
normalização conservadora das variáveis de entrada para escala
{[}0.10{]}, com inversão aplicada à densidade do solo dada sua relação
inversa com qualidade física; a definição de funções de pertinência
triangulares e regras linguísticas baseadas em conhecimento
especializado; e a defuzzificação dos valores por meio do método do
centróide, resultando em um valor contínuo de desempenho funcional do
solo. As variáveis de entrada selecionadas representaram frações
funcionais da matéria orgânica do solo e atributos físicos, incluindo N
total, P total (valores médios de concentrações e estoques de nitrogênio
e fósforo), e densidade do solo (Ds) como variável limitante associada à
degradação física. Adicionalmente, os teores de areia, silte e argila
foram considerados covariáveis de controle. Todas as variáveis foram
derivadas de um conjunto mais amplo de atributos edáficos medidos em
áreas com diferentes históricos de uso e cobertura da terra.

\subsubsection{2.8.3 Normalização das
variáveis}\label{normalizauxe7uxe3o-das-variuxe1veis}

Para garantir a comparabilidade entre variáveis de diferentes escalas e
ordens de grandeza, os dados foram normalizados em uma escala comum de 0
a 10. Para padronizar as variáveis em uma escala comum entre 0 e 10,
adotaram-se dois modelos de normalização linear conservadora (Equação
1). x\_norm=10* (x- x\_min)/(x\_max- x\_min ) ,com x ∈ {[} x\_min-
x\_max{]} Em que, x: valor original da variável xmin é o menor valor
observado da variável x, xmax é o maior valor observado da variável x e
xnorm sendo o valor normalizado na escala de 0 a 10

As variáveis com interpretação inversa (como a densidade do solo, cuja
elevação está associada a menor qualidade estrutural), foi aplicada uma
normalização invertida (Equação 2), atribuindo notas mais altas a
valores menores:

x\_norm=10* (x\_max- x)/(x\_max- x\_min ) ,com x ∈ {[} x\_min- x\_max{]}
Que, Quando x = xmin, então xnorm =10 → melhor valor. Quando x = xmax,
então xnorm =0 → pior valor.

Em ambos os casos, valores não finitos ou constantes foram tratados com
imputações neutras (nota 5), assegurando robustez numérica.

\subsubsection{2.8.4 Estrutura do sistema fuzzy e regras de
inferência}\label{estrutura-do-sistema-fuzzy-e-regras-de-inferuxeancia}

O sistema de inferência fuzzy foi implementado no ambiente R por meio do
pacote FuzzyR, adotando-se a lógica do tipo Mamdani para construção do
Índice de Sustentação Nutricional Fuzzy (FSNSI). Foram utilizados três
variáveis de entrada: nitrogênio total normalizado (N\_total), fósforo
total normalizado (P total) e densidade do solo normalizada e invertida
(Ds). A variável de saída FSNSI, foi modelada com três termos
linguísticos qualitativos: baixa, média e alta, todos definidos no
intervalo contínuo de 0 a 10. As funções de pertinência seguem a forma
geral de triângulos simétricos ou assimétricos, conforme Equação 3 para
termo Baixo, Equação 4 termo Médio e Equação 5 para o termo Alto.

μbaixa (x)=max⁡(min((x-a)/(b-a),(c-x)/(c-b)),0),com(a,b,c)=(0,0,4)

μmmédia (x)=max⁡(min((x-a)/(b-a),(c-x)/(c-b)),0),com(a,b,c)=(3,5,7)

μmalta (x)=max⁡(min((x-a)/(b-a),(c-x)/(c-b)),0),com(a,b,c)=(6,10,10)

O domínio da função triangular foi ajustado de forma a garantir
sobreposição entre os termos, permitindo inferências contínuas e
suavizadas.

\subsubsection{2.8.5 Base de regras fuzzy}\label{base-de-regras-fuzzy}

A base de conhecimento foi composta por sete regras linguísticas do tipo
``SE--ENTÃO'', baseadas na combinação dos termos linguísticos das
variáveis de entrada. A estrutura das regras segue a forma genérica
(Equação 6):

R\_(i ):SE x\_1 ∈ A\_1 ∧ x\_2 ∈ A\_2 ∧ x\_3 ∈ A\_3⇒y ∈ B\_i

A ativação de cada regra R\_(i ) foi calculada conforme a Equação 7:

\[\alpha_i = \min(\mu_{A1}(x_1), \mu_{A2}(x_2), \dots, \mu_{An}(x_n))\]

O valor de saída correspondente μ\_Bi (z) foi ponderado por αi e os
conjuntos ativados foram agregados via operador do tipo MAX. Para
incorporar o papel da densidade do solo como penalidade funcional, os
pesos wi atribuídos a cada regra foram ajustados dinamicamente em função
da classificação linguística de Ds. Regras com densidade alta (categoria
linguística ``baixa'' na escala invertida) receberam pesos reduzidos
(entre 0.4 e 0.6), enquanto regras com densidade baixa (categoria
``alta'') foram atribuídas com pesos maiores (até 1.5). Esse ajuste visa
reforçar a influência da qualidade física do solo na determinação da
funcionalidade.

\subsubsection{2.8.6 Defuzzificação do índice
funcional}\label{defuzzificauxe7uxe3o-do-uxedndice-funcional}

O valor final do índice FSNSI foi obtido por meio do processo de
defuzzificação por centróide, conforme Equação 8: FSNSI=
(∫\_0\^{}10〖μFSNSI (z)*zdz〗)/(∫\_0\^{}10〖μFSNSI (z)*dz〗)

Em que μFSNSI(z) representa a função de pertinência agregada resultante
da combinação de todas as regras ativadas para um determinado conjunto
de entradas, e (z) é o domínio contínuo da variável de saída. O valor
crisp resultante do índice FSNSI assume valores contínuos entre 0 e 10 e
foi interpretado segundo três faixas qualitativas sendo (0.0 a 3.3)
funcionalidade baixa; (3.4 a 6.6) funcionalidade intermediária e (6.7 a
10.0) funcionalidade alta.

\bookmarksetup{startatroot}

\chapter{Importância das frações de N e P para predição dos estoques
totais
(PLSR)}\label{importuxe2ncia-das-frauxe7uxf5es-de-n-e-p-para-prediuxe7uxe3o-dos-estoques-totais-plsr}

\# Resultados e Discussão

A análise de regressão por mínimos quadrados parciais (PLSR) revelou que
as frações de nitrogênio e fósforo apresentam contribuições
diferenciadas para a predição dos estoques totais desses nutrientes no
solo. A análise dos escores de importância das variáveis (VIP)
identificou as frações mais relevantes para explicar a variabilidade de
NT e PT nos sistemas estudados.

\subsection{Predição do Nitrogênio
Total}\label{prediuxe7uxe3o-do-nitroguxeanio-total}

A análise PLSR para o nitrogênio total (NT) foi realizada utilizando as
frações de N como variáveis preditoras, permitindo identificar as
contribuições relativas de cada compartimento funcional. Os resultados
são apresentados na Figura 4, que mostra o biplot com os scores das
amostras por uso da terra e os loadings das variáveis preditoras,
evidenciando a estrutura de covariância entre as frações e os estoques
totais. Os resultados demonstram que as frações húmicas NTHum e NTAH são
as preditoras mais importantes (VIP \textgreater{} 1.0), indicando sua
dominância na explicação dos estoques totais de N. As frações lábeis
NLabil e NMOL também contribuem significativamente (VIP 0.8-1.0),
enquanto NTAF apresenta relevância moderada (Figura 4). Esse padrão
reflete a importância das frações húmicas como reservatórios estáveis de
N em solos tropicais, corroborando estudos que destacam seu papel na
sustentabilidade nutricional (Paul, 2016).

A Tabela 2 sintetiza os escores de importância na projeção (VIP) obtidos
a partir do modelo PLSR. O modelo foi estimado com duas componentes
latentes, que explicaram mais de 80\% da variância acumulada entre
preditores e resposta, com validação cruzada leave‑one‑out para checagem
da robustez preditiva. As frações húmicas especialmente NTHum (VIP ≈
1.2--1.5) e NTAH (VIP ≈ 1.1--1.3), apresentaram os maiores valores de
VIP, caracterizando‑se como preditores críticos para a variabilidade do
nitrogênio total (NT). As frações lábeis (NLabil e NMOL) exibiram VIP
próximos a 0.8--1.0, indicando contribuição relevante porém secundária,
enquanto NTAF mostrou relevância moderada (VIP \textless{} 0.8).

Esses resultados são congruentes com a interpretação funcional em que as
frações húmicas, por possuírem maior massa molar, maior associação
organo‑mineral e resistência à mineralização, atuam como reservatórios
de longo prazo que determinam os estoques totais de N; já as frações
lábeis refletem pools de rápida renovação e resposta a entradas recentes
de matéria orgânica e manejo. No biplot (Figura 4) observa‑se segregação
espacial das amostras por uso da terra, com agrupamentos associados a
maiores loadings húmicos (por exemplo, Eucalipto e Cerrado) e outros
mais voltados ao eixo de frações lábeis (Agricultura e Pastagem), o que
corrobora a hipótese de que a qualidade da serapilheira e o manejo
estrutural modulam a composição funcional do pool de N.

Do ponto de vista gerencial e ecológico, a predominância das frações
húmicas na predição do NT indica que práticas voltadas ao aumento da
estabilidade da MOS (redução do revolvimento, incremento de inputs
lignificados, manutenção de cobertura morta) tendem a promover retenção
de N em formas de maior persistência, fortalecendo a capacidade de
armazenamento e a resiliência do sistema. Em contrapartida, sistemas com
maior proporção de frações lábeis apresentam maior disponibilidade
imediata de N, porém menor capacidade de retenção a longo prazo,
implicando necessidade de intervenções de manejo contínuo para
manutenção da fertilidade.

\subsection{3.1.2 Predição do Fósforo
Total}\label{prediuxe7uxe3o-do-fuxf3sforo-total}

De forma similar ao observado para o nitrogênio, os teores de P total
(PT) e suas frações húmicas (PTHum, PTAH, PLabil) apresentaram elevada
correlação estrutural no modelo PLSR (Figura 5). Esse comportamento está
alinhado ao conceito de co-acúmulo e co-estabilização de nutrientes na
MOS, segundo o qual P e N se associam a complexos organo-minerais ou são
adsorvidos simultaneamente a colóides orgânicos e minerais, favorecendo
a persistência dos estoques totais (Tivet et al., 2013, Helfenstein et
al., 2020). A aplicação de técnicas multivariadas como PLSR em estudos
de ciência do solo tem se mostrado particularmente adequada para lidar
com conjuntos de dados caracterizados por multicolinearidade entre
variáveis, conforme demonstrado por Sekaran et al. (2020) em análises de
atividades bioquímicas e estrutura de comunidades microbianas do solo.

A correlação entre PLabil/PMOL e as frações húmicas de P sugere sinergia
entre disponibilidade de nutrientes e fracionamento da matéria orgânica,
indicando processos dinâmicos de transformação e estabilização do
fósforo no solo. Os mecanismos de co-estabilização envolvem (i) adsorção
específica de ortofosfato (H₂PO₄⁻) em superfícies de ácidos húmicos via
pontes de hidrogênio e complexação com grupos funcionais carboxílicos e
fenólicos, (ii) formação de complexos ternários envolvendo P orgânico,
cátions polivalentes (Ca²⁺, Fe³⁺, Al³⁺) e substâncias húmicas, que
reduzem a solubilidade e a mobilidade do fósforo no perfil, e (iii)
oclusão física de P lábil em microagregados estabilizados por humina e
material recalcitrante, protegendo-o contra mineralização rápida. Esses
processos são particularmente relevantes em Latossolos altamente
intemperizados, onde a mineralogia dominada por caulinita, gibbsita e
óxidos de Fe/Al apresenta elevada capacidade de fixação de P, e a MOS
atua como moduladora dessa fixação através de competição por sítios de
adsorção.

\includegraphics[width=0.9\linewidth,height=\textheight,keepaspectratio]{chapters/../../2-FIGURAS/biplot_plsr_pt.png}
Nota: Padrão visual similar ao do NT, refletindo a associação estrutural
entre N e P no solo.

Os escores VIP para as frações de P revelaram padrão similar ao do
nitrogênio, com PTHum e PTAH como preditores dominantes (Tabela 5). Esse
paralelismo estrutural entre N e P reforça que os processos de
humificação progressiva---onde compostos lábeis são gradativamente
convertidos em frações húmicas intermediárias (ácidos fúlvicos → ácidos
húmicos) e finalmente em humina recalcitrante---operam de forma
sincronizada para ambos os nutrientes. A predominância de PTHum como
preditor-chave reflete o fato de que, em solos tropicais sob vegetação
estabelecida, a maior parte do P orgânico encontra-se imobilizada em
formas estáveis de longa persistência (décadas a séculos), contrastando
com sistemas agrícolas onde o revolvimento frequente acelera a
mineralização e favorece o acúmulo relativo de frações lábeis. A
correlação significativa entre VIP(PTHum) e VIP(NTHum) (r \textgreater{}
0.85) evidencia que os mesmos fatores edafoclimáticos---temperatura,
umidade, atividade microbiana, textura do solo---regulam simultaneamente
o fracionamento de P e N, sugerindo que estratégias de manejo
direcionadas à conservação de frações húmicas beneficiam ambos os
nutrientes de forma integrada.

\begin{longtable}[]{@{}lll@{}}
\caption{Tabela 5 - Escores VIP das variáveis preditoras de PT (Variável
Importance in Projection). VIP \textgreater{} 1.0 indica variáveis
críticas para a predição do PT.}\tabularnewline
\toprule\noalign{}
Variável & VIP & Relevância \\
\midrule\noalign{}
\endfirsthead
\toprule\noalign{}
Variável & VIP & Relevância \\
\midrule\noalign{}
\endhead
\bottomrule\noalign{}
\endlastfoot
PTHum & \textgreater{} 1.0 & Muito importante \\
PTAH & \textgreater{} 1.0 & Muito importante \\
PLabil & 0.8-1.0 & Importante \\
PMOL & 0.8-1.0 & Importante \\
PTAF & \textless{} 0.8 & Moderada \\
\end{longtable}

Esses achados reforçam que a predição de NT e PT em sistemas edáficos
tropicais deve considerar não apenas frações diretamente associadas a
cada elemento, mas também a natureza multinutriente da MOS e sua
co-regulação pelas vias biogeoquímicas de decomposição, complexação e
proteção físico-química. A elevada capacidade preditiva do modelo PLSR
(variância explicada \textgreater{} 80\% com duas componentes latentes)
demonstra que o fracionamento químico capta efetivamente os processos
funcionais que governam a disponibilidade de nutrientes, fornecendo base
sólida para desenvolvimento de indicadores de qualidade do solo e
monitoramento de trajetórias de degradação ou recuperação em diferentes
sistemas de uso da terra.

\section*{Relações estruturais entre frações lábeis, húmicas e estoques
totais
(PLS-SEM)}\label{relauxe7uxf5es-estruturais-entre-frauxe7uxf5es-luxe1beis-huxfamicas-e-estoques-totais-pls-sem}
\addcontentsline{toc}{section}{Relações estruturais entre frações
lábeis, húmicas e estoques totais (PLS-SEM)}

\markright{Relações estruturais entre frações lábeis, húmicas e estoques
totais (PLS-SEM)}

\subsection{3.2.1 Resultados do modelo estrutural
global}\label{resultados-do-modelo-estrutural-global}

O modelo estrutural foi configurado como um modelo de componentes
hierárquicos (hierarchical component model, HCM) em dois níveis,
combinando construtos de primeira ordem que representam as frações
funcionais de nitrogênio e fósforo e construtos de segunda ordem que
sintetizam os estoques totais. Essa abordagem segue as recomendações de
Hair et al. (2021) para modelagem de estruturas reflexivas com
indicadores repetidos, assegurando que a variância compartilhada entre
as dimensões funcionais seja propagada para os construtos superiores. No
primeiro nível, N\_lábil e P\_lábil foram medidos pelos pares
NLabil/NMOL e PLabil/PMOL, enquanto N\_húmico e P\_húmico agregaram
NTAF, NTAH, NTHum e PTAF, PTAH, PTHum, respectivamente. A etapa de
mensuração revelou cargas fatoriais elevadas (λ = 0.981 a 0.990) para
todos os indicadores, confirmando que as variáveis manifestas capturam
de forma robusta a variabilidade intrínseca de cada fração funcional
(Figura 6).

\includegraphics[width=0.95\linewidth,height=\textheight,keepaspectratio]{chapters/../../2-FIGURAS/analise_caminhos_pls.png}
Nota: As setas indicam as trajetórias de influência causal, com largura
proporcional à magnitude dos coeficientes de caminho. O modelo revela
que as frações húmicas de N e P apresentam maior poder preditivo dos
estoques totais comparativamente às frações lábeis, com coeficientes de
caminho variando entre ambientes de estudo, demonstrando heterogeneidade
funcional induzida pelo manejo.

A consistência interna elevada (α \textgreater{} 0.97) e a variância
média extraída acima de 0.95 (Tabela 4) indicam confiabilidade
convergente excelente para todos os construtos, superando com folga os
limiares recomendados para estudos com estruturas reflexivas (Hair et
al., 2021). A inspeção da matriz de cargas cruzadas e da razão HTMT
confirmou ausência de problemas de validade discriminante, demonstrando
que as frações lábeis e húmicas retêm especificidades conceituais mesmo
quando agregadas em construtos de segunda ordem. Dessa forma, a etapa de
mensuração oferece base robusta para interpretar os resultados
estruturais, minimizando o risco de vieses associados a colinearidade
excessiva entre indicadores.

Na etapa estrutural, o modelo global apresentou coeficiente de
determinação de 0.959 tanto para N\_total quanto para P\_total,
explicando praticamente toda a variabilidade observada dos estoques em
condições de Cerrado (Tabela 4). Esse nível de ajuste é raro em estudos
edáficos e indica que o fracionamento funcional resume a maior parte dos
processos que modulam o acúmulo de N e P. Os coeficientes de caminho
oriundos das frações húmicas (β = 1.286 para N\_húmico → N\_total e
P\_húmico → P\_total) foram positivos, elevados e estatisticamente
diferentes de zero nas reamostragens bootstrap, evidenciando que
acréscimos marginais nesses compartimentos resultam em aumentos
proporcionais superiores nos estoques totais. Esse comportamento é
coerente com evidências de campo que apontam as frações húmicas como
reservatórios dominantes de nutrientes em Latossolos sob silvicultura e
pastagens de longa duração (Pegoraro et al., 2011, Ferreira et al.,
2021).

\begin{longtable}[]{@{}lll@{}}
\caption{Tabela 4 - Métricas de qualidade do modelo PLS-SEM global.
Nota: Os valores indicam adequação do modelo aos dados empíricos
conforme critérios de referência da metodologia.}\tabularnewline
\toprule\noalign{}
Métrica & Valor Observado & Interpretação \\
\midrule\noalign{}
\endfirsthead
\toprule\noalign{}
Métrica & Valor Observado & Interpretação \\
\midrule\noalign{}
\endhead
\bottomrule\noalign{}
\endlastfoot
R² (N\_total) & 0.959 (Cerrado) & Explicação de 95.9\% da variância \\
R² (P\_total) & 0.959 (Cerrado) & Explicação de 95.9\% da variância \\
SRMR & \textless{} 0.06 & Modelo bem ajustado aos dados \\
Confiabilidade (α) & \textgreater{} 0.97 & Consistência interna
\textgreater{} 0.97 \\
Comunalidade (AVE) & \textgreater{} 0.95 & Variância média bem
explicada \\
\end{longtable}

Os coeficientes negativos associados às frações lábeis (β = -0.313 para
N\_lábil → N\_total e P\_lábil → P\_total) refletem o mecanismo de
particionamento entre pools de rápida renovação e reservas estáveis. À
medida que o sistema acumula compostos recalcitrantes, parte das formas
lábeis é incorporada à matriz húmica, produzindo efeito inverso nos
modelos estruturais. Esse resultado tem respaldo em cronossequências que
documentam a transferência progressiva de N e P lábeis para frações
humificadas durante a maturação de sistemas florestais e agrícolas
(Marinho Junior et al., 2021, Araújo Filho et al., 2024). Não se trata,
portanto, de antagonismo biogeoquímico, mas de evidência de que o modelo
capturou a dinâmica de humificação incremental do sistema.

A simetria dos coeficientes entre N e P reforça a hipótese de
acoplamento biogeoquímico dos dois macronutrientes, governado por
processos comuns de humificação, complexação organo-mineral e proteção
estrutural da matéria orgânica. Esse padrão corrobora observações em
agroflorestas de café sombreado no Nordeste, onde o incremento do
carbono orgânico total está associado simultaneamente a aumentos nas
frações húmicas de N e P e à melhoria da estrutura física do solo
(Crespo et al., 2024). Os resultados do PLS-SEM confirmam que a
estabilização conjunta de N e P é um atributo-chave de sistemas
manejados com alta entrada de resíduos lignificados, elemento central
para compreender a funcionalidade edáfica nos usos da terra avaliados.

A modelagem por equações estruturais baseada em mínimos quadrados
parciais (PLS-SEM) permitiu quantificar as contribuições relativas das
frações lábeis e húmicas de N e P para os estoques totais desses
nutrientes. O resultado mais significativo foi o alcance de R² = 0.959
para ambos N\_total e P\_total no ambiente Cerrado, indicando que o
modelo explica 95.9\% da variância observada. Esse coeficiente de
determinação representa performance excepcional em modelagem de sistemas
edáficos complexos, especialmente considerando a heterogeneidade
inerente aos ecossistemas de solo tropical, caracterizados por
variabilidade espacial elevada e múltiplos mecanismos de transformação
de nutrientes. A replicação desse desempenho entre N e P (R² = 0.959
para ambos) evidencia que os mesmos compartimentos funcionais (frações
húmicas e lábeis) operem como determinantes estruturais para ambos os
macronutrientes, sugerindo acoplamento geoquímico estreito entre as
dinâmicas de N e P nessa matriz de solo.

O diagrama do modelo (Figura 6) representa graficamente essa
arquitetura, onde retângulos indicam variáveis manifestas (indicadores
observáveis), hexágonos representam construtos latentes (variáveis não
observadas diretamente) e setas mostram as relações causais hipotéticas.
As cargas fatoriais elevadas (λ = 0.981-0.990 para todos os indicadores)
confirmam a qualidade excepcional da medição das variáveis latentes,
validando a escolha dos indicadores para representação das frações de
nutrientes.

A relação positiva dominante das frações húmicas (β = 1.286) corrobora a
hipótese de que essas frações constituem o principal reservatório de
longo prazo desses nutrientes, representando aproximadamente 56\% a mais
da contribuição unitária para os estoques totais comparativamente aos
terços finais do modelo. A magnitude de β = 1.286 indica que incrementos
de uma unidade na fração húmica geram aumento de 1.286 unidades no
estoque total, estabelecendo essas frações como preditoras dominantes da
disponibilidade de nutrientes no sistema. Essa dominância das frações
húmicas é consistente com observações de Pegoraro et al. (2011), que
demonstraram que substâncias húmicas (ácidos fúlvicos, húmicos e
huminas) representam os principais compartimentos de C e N em solos sob
diferentes usos da terra, e com Ferreira et al. (2021), que evidenciaram
que a matéria orgânica associada aos minerais (MAOM) constitui o
reservatório mais estável de nutrientes em solos tropicais brasileiros.

Estudos em sistemas agroflorestais sombreados confirmam esse padrão,
demonstrando que a presença de árvores nativas modula significativamente
os processos biogeoquímicos do solo. Crespo et al. (2024) evidenciaram
em sistemas de café sombreado que o carbono orgânico total (COT) atua
como integrador central entre porosidade, acidez e disponibilidade de
nutrientes, sendo que áreas com maior cobertura arbórea apresentaram COT
superior a 2.5\% e microporosidade de \textasciitilde30\%, refletindo o
acúmulo coluvial e a proteção contra a mineralização acelerada. A
análise de componentes principais nesses sistemas explicou 60.2\% da
variação total, com PC1 (estrutura física e fertilidade, 38.5\%) e PC2
(qualidade da matéria orgânica, 21.7\%), corroborando a importância das
frações húmicas como determinantes estruturais da funcionalidade
edáfica. Esses resultados reforçam que a integração entre diversidade
arbórea e processos pedogenéticos favorece a estabilização de nutrientes
nas frações húmicas, processo particularmente relevante em Latossolos
altamente intemperizados onde a baixa capacidade de troca catiônica
amplia a dependência da matéria orgânica para retenção de N e P.

A relação negativa das frações lábeis (β = -0.313), embora aparentemente
contraditória à primeira vista, reflete um mecanismo biogeoquímico
fundamental em solos tropicais: a colinearidade estrutural representada
no modelo constitui transformação matemática da interação dinâmica entre
pools. Especificamente, quando a fração húmica aumenta, a fração lábil
diminui relativamente, pois o modelo captura o fenômeno de humificação
progressiva onde compartimentos lábeis são continuamente convertidos em
frações húmicas recalcitrantes (Marinho Junior et al., 2021). Essa
dinâmica não indica relação causal negativa entre os compartimentos, mas
sim descreve a natureza do particionamento de nutrientes: em solos com
elevada decomposição microbiana (como o Cerrado), a conversão N\_lábil →
N\_húmico → N\_recalcitrante ocorre sequencialmente, resultando em
compensação de sinais nos coeficientes de regressão múltipla. O padrão
β(húmico) = +1.286 e β(lábil) = -0.313 mantém proporcionalidade
esperada, onde N\_total = 1.286×N\_húmico - 0.313×N\_lábil apresenta
ajuste estatístico que evita multicolinearidade excessiva.

Essa interpretação encontra suporte direto na observação dos dados: se
as frações húmicas fossem verdadeiramente prejudiciais (como indicaria
um β negativo), sistemas com elevada humificação não apresentariam
acúmulo consistente de nutrientes. Inversamente, o padrão observado
confirma que as frações húmicas exercem efeito funcional positivo,
validando a interpretação de que os coeficientes negativos no PLS-SEM
refletem estrutura matemática de compensação em vez de antagonismo
biogeoquímico.

\subsection{3.2.3 Resultados da análise multigrupo por uso da terra
(PLS-MGA)}\label{resultados-da-anuxe1lise-multigrupo-por-uso-da-terra-pls-mga}

A Figura 7 evidencia que a análise multigrupo (PLS-MGA) expôs
heterogeneidade marcante na arquitetura estrutural entre os usos da
terra, com coeficientes de caminho modulados pelo manejo e pela biomassa
depositada. O Cerrado nativo sustenta o maior efeito húmico entre os
sistemas naturais (β = 1.286), configurando a referência biogeoquímica
de humificação equilibrada na qual acúmulo e mineralização operam em
dinâmica quasi-estacionária. Esse patamar delineia a linha de base usada
para interpretar os desvios observados nos demais ambientes. A
agricultura convencional desloca-se dessa referência ao apresentar
β(húmico) = 1.180, 8.2\% inferior ao Cerrado. A redução indica
reconfiguração do particionamento entre pools: a menor contribuição
húmica é acompanhada por fração lábil menos negativa (-0.237 contra
-0.313 no Cerrado), sinalizando que o revolvimento mecânico e a entrada
recorrente de fertilizantes minerais mantêm proporção elevada de
compostos rapidamente disponíveis. Em sistemas agrícolas tropicais, tais
práticas aceleram a transferência de N e P das reservas estáveis para
formas solúveis, tornando indispensável a adubação de manutenção para
evitar quedas abruptas nos estoques (Santos et al., 2024).

\begin{figure}

{\centering \includegraphics[width=0.95\linewidth,height=\textheight,keepaspectratio]{chapters/../../2-FIGURAS/comparacao_coeficientes.png}

}

\caption{Figura 7. Comparação dos coeficientes de caminho entre os
diferentes usos da terra. Barras agrupadas mostram a magnitude dos
efeitos (N\_húmico → N\_total e P\_húmico → P\_total) para cada
ambiente. Cores distintas representam os cinco sistemas de uso
avaliados.}

\end{figure}%

Os sistemas silviculturais ocupam faixa intermediária (Tabela 5) em que,
o Mogno-africano (β = 1.271) conserva estrutura similar ao Cerrado,
coerente com regime de manejo menos intensivo e serapilheira de
decomposição moderada. O Eucalipto (β = 1.275) supera ligeiramente o
Mogno, refletindo acúmulo expressivo de resíduos lignificados que
favorecem a incorporação de N e P nas frações húmicas, fenômeno já
documentado para plantações maduras da espécie (Pegoraro et al., 2011,
Ferreira et al., 2021). A Teca, contudo, apresenta o coeficiente máximo
(β = 1.372), indicando humificação intensificada. Embora à primeira
vista pareça indicativo de maior estabilidade, o valor elevado ocorre em
conjunto com estoques totais mais baixos e com densidade aparente
elevada, sugerindo que a serapilheira recalcitrante da Teca retarda a
ciclagem e concentra nutrientes em compartimentos pouco acessíveis. A
progressão Cerrado → Mogno → Eucalipto → Teca reforça que a qualidade da
serapilheira é o principal vetor de diferenciação entre os sistemas.
Serapilheiras com alta relação C:N e grande teor de lignina, como as de
Tectona grandis, tendem a formar compostos resistentes que privilegiam o
enriquecimento húmico em detrimento das frações lábeis, reduzindo a
velocidade de reciclagem e a disponibilidade imediata (Araújo Filho et
al., 2024). Em contraposição, resíduos com maior labilidade (por
exemplo, consórcios leguminosos) alimentam o pool lábil e mantêm a
renovação rápida dos nutrientes, condição que pode ser obtida pela
introdução de adubação verde e pela diversificação de espécies arbóreas
nos talhões.

Para o intervalo de variação de β(húmico) de 1.180 (Agricultura) a 1.372
(Teca) representa amplitude relativa de 16.2\%, magnitude expressiva
considerando que o modelo já explica 95.9\% da variância em cenários
ótimos. Tal amplitude traduz-se em recomendações práticas: (i) manter β
próximo a 1.27 nos sistemas conservacionistas (Cerrado, Mogno) por meio
da proteção do horizonte superficial; (ii) sustentar programas de
adubação e cobertura morta na Agricultura para evitar colapso do pool
húmico; e (iii) adotar estratégias de descompactação e enriquecimento
com espécies de baixa relação C:N nas plantações de Teca, uma vez que o
coeficiente elevado não compensa as restrições físicas que limitam a
disponibilização dos nutrientes.

\begin{longtable}[]{@{}
  >{\raggedright\arraybackslash}p{(\linewidth - 8\tabcolsep) * \real{0.1556}}
  >{\raggedright\arraybackslash}p{(\linewidth - 8\tabcolsep) * \real{0.2111}}
  >{\raggedright\arraybackslash}p{(\linewidth - 8\tabcolsep) * \real{0.2111}}
  >{\raggedright\arraybackslash}p{(\linewidth - 8\tabcolsep) * \real{0.2111}}
  >{\raggedright\arraybackslash}p{(\linewidth - 8\tabcolsep) * \real{0.2111}}@{}}
\caption{Tabela 5 - Coeficientes de caminho estratificados por uso da
terra. Valores padronizados indicam a magnitude das relações estruturais
em cada ambiente. Valores positivos indicam efeitos diretos; negativos,
efeitos inversos ou ajustes de colinearidade.}\tabularnewline
\toprule\noalign{}
\begin{minipage}[b]{\linewidth}\raggedright
Uso da Terra
\end{minipage} & \begin{minipage}[b]{\linewidth}\raggedright
N\_húmico → N\_total
\end{minipage} & \begin{minipage}[b]{\linewidth}\raggedright
N\_lábil → N\_total
\end{minipage} & \begin{minipage}[b]{\linewidth}\raggedright
P\_húmico → P\_total
\end{minipage} & \begin{minipage}[b]{\linewidth}\raggedright
P\_lábil → P\_total
\end{minipage} \\
\midrule\noalign{}
\endfirsthead
\toprule\noalign{}
\begin{minipage}[b]{\linewidth}\raggedright
Uso da Terra
\end{minipage} & \begin{minipage}[b]{\linewidth}\raggedright
N\_húmico → N\_total
\end{minipage} & \begin{minipage}[b]{\linewidth}\raggedright
N\_lábil → N\_total
\end{minipage} & \begin{minipage}[b]{\linewidth}\raggedright
P\_húmico → P\_total
\end{minipage} & \begin{minipage}[b]{\linewidth}\raggedright
P\_lábil → P\_total
\end{minipage} \\
\midrule\noalign{}
\endhead
\bottomrule\noalign{}
\endlastfoot
Cerrado & 1.286 & -0.313 & 1.286 & -0.313 \\
Agricultura & 1.180 & -0.237 & 1.180 & -0.237 \\
Mogno-africano & 1.271 & -0.277 & 1.271 & -0.277 \\
Eucalipto & 1.275 & -0.283 & 1.275 & -0.283 \\
Teca & 1.372 & -0.445 & 1.372 & -0.445 \\
\end{longtable}

\section*{Integração fuzzy da funcionalidade edáfica: Índice de
Sustentabilidade
(FSNSI)}\label{integrauxe7uxe3o-fuzzy-da-funcionalidade-eduxe1fica-uxedndice-de-sustentabilidade-fsnsi}
\addcontentsline{toc}{section}{Integração fuzzy da funcionalidade
edáfica: Índice de Sustentabilidade (FSNSI)}

\markright{Integração fuzzy da funcionalidade edáfica: Índice de
Sustentabilidade (FSNSI)}

\subsection{3.3.1 Sistema de inferência fuzzy e funções de
pertinência}\label{sistema-de-inferuxeancia-fuzzy-e-funuxe7uxf5es-de-pertinuxeancia}

O Índice Fuzzy de Sustentabilidade Edáfica (FSNSI) foi construído a
partir de um sistema de inferência fuzzy do tipo Mamdani, integrando
três variáveis de entrada (N total, P total e densidade do solo) e uma
variável de saída (FSNSI, escala 0-10). As variáveis de entrada foram
normalizadas para escala 0-10, com densidade do solo recebendo inversão
matemática para refletir sua relação negativa com a funcionalidade
edáfica. O sistema fuzzy utilizou funções de pertinência triangulares
para representar três classes linguísticas (baixa, média e alta) em cada
variável de entrada, e três classes na saída (baixa: 0-3.33, média:
3.34-6.66, alta: 6.67-10.0).

A Figura 7 ilustra as funções de pertinência para as variáveis de
entrada, evidenciando a estrutura lógica do sistema fuzzy. As regiões de
sobreposição entre classes permitem transições graduais entre estados
linguísticos, característica fundamental da lógica fuzzy que distingue
essa abordagem de classificações discretas convencionais. Para N total e
P total, valores normalizados abaixo de 3.3 foram classificados como
``baixos'', entre 3.3 e 6.7 como ``médios'', e acima de 6.7 como
``altos''. A densidade do solo foi categorizada como ``baixa''
(\textless{} 1.2 g/cm³), ``média'' (1.2-1.4 g/cm³) e ``alta''
(\textgreater{} 1.4 g/cm³).

\begin{figure}

{\centering \includegraphics[width=0.9\linewidth,height=\textheight,keepaspectratio]{chapters/../../2-FIGURAS/funcoes_pertinencia_entrada.png}

}

\caption{Figura 7. Funções de pertinência fuzzy para as variáveis de
entrada (N total, P total e densidade do solo) do sistema de inferência
Mamdani utilizado no cálculo do FSNSI. As regiões de sobreposição
permitem transições graduais entre classes linguísticas (baixa, média,
alta).}

\end{figure}%

A Figura 8 apresenta a função de pertinência da variável de saída
(FSNSI), estruturada em três categorias de funcionalidade edáfica: baixa
(0-3.33, indicando degradação severa), média (3.34-6.66, funcionalidade
moderada) e alta (6.67-10.0, sustentabilidade plena). O sistema foi
configurado com 16 regras de inferência (2³ = 8 regras base, duplicadas
para considerar assimetrias nas combinações de entrada), seguindo a
estrutura lógica: ``SE N total é X E P total é Y E densidade é Z, ENTÃO
FSNSI é W''. As regras foram definidas com base em conhecimento
pedológico especializado, priorizando combinações sinérgicas entre alta
disponibilidade química de nutrientes e baixa compactação. O método de
defuzzificação adotado foi o centroide, que calcula o valor crisp de
saída como o centro de gravidade da função de pertinência resultante da
agregação das regras ativadas.

\begin{figure}

{\centering \includegraphics[width=0.8\linewidth,height=\textheight,keepaspectratio]{chapters/../../2-FIGURAS/funcao_pertinencia_saida.png}

}

\caption{Figura 8. Função de pertinência fuzzy para a variável de saída
FSNSI (Fuzzy Soil Nutrient Sustainability Index). Três classes
linguísticas representam níveis de funcionalidade edáfica: baixa
(0-3.33), média (3.34-6.66) e alta (6.67-10.0).}

\end{figure}%

Ainda, a análise do FSNSI revelou diferenciação significativa entre os
sistemas de uso da terra avaliados (Figura 9, Tabela 6). O Eucalipto
apresentou o maior FSNSI médio (6.07 ± 2.78, n = 40), classificado
predominantemente como funcionalidade ``alta'' (60\% das amostras),
seguido por Mogno-africano (4.54 ± 2.89), Cerrado nativo (4.42 ± 2.89),
Agricultura convencional (4.25 ± 2.57) e Teca (2.77 ± 1.16), este último
com funcionalidade predominantemente ``baixa'' (62.5\% das amostras). A
análise de variância (ANOVA) confirmou diferenças significativas entre
os usos (F = 12.84, p \textless{} 0.001), com o teste de Tukey HSD
identificando três agrupamentos estatísticos distintos.

\begin{figure}

{\centering \includegraphics[width=0.9\linewidth,height=\textheight,keepaspectratio]{chapters/../../2-FIGURAS/boxplot_fsnsi_uso_terra.png}

}

\caption{Figura 9. Distribuição do Índice Fuzzy de Sustentabilidade
Edáfica (FSNSI) por uso da terra. Boxplots com padrões visuais distintos
para cada sistema. Letras diferentes indicam diferenças significativas
pelo teste de Tukey HSD (p \textless{} 0.05). Pontos representam valores
individuais.}

\end{figure}%

O desempenho do Eucalipto (FSNSI = 6.07) diferencia-se
significativamente do Cerrado nativo (FSNSI = 4.42), com diferença
estatisticamente significativa (Δ = 1.64, IC 95\%: 0.07-3.22, p =
0.036). Esse resultado contraria a expectativa inicial de que a
vegetação nativa representaria o estado de funcionalidade máxima. A
explicação reside na combinação entre: elevado aporte de serapilheira
rica em lignina e compostos fenólicos pelo Eucalipto, favorecendo a
formação de frações húmicas estáveis de N e P; sistema radicular
profundo que promove bioperturbação do solo, reduzindo localmente a
densidade aparente nas camadas superficiais; e ausência de remoção de
biomassa por colheita no sistema avaliado (povoamento não manejado há 11
anos), permitindo acúmulo contínuo de matéria orgânica. Silva et al.
(2024) documentaram que a conversão de áreas nativas de Cerrado para
sistemas agrícolas altera profundamente a dinâmica de emissão de CO₂ e
os estoques de carbono no solo, enquanto Leal et al. (2024) demonstraram
que plantios de Acacia e Eucalyptus modificam a composição molecular das
frações de matéria orgânica particulada (POM) e associada aos minerais
(MAOM) em solos de pastagens nativas subtropicais, sugerindo que a
qualidade dos resíduos vegetais exerce papel determinante na
estabilização de nutrientes. O Cerrado nativo apresentou FSNSI médio de
4.42, classificado como funcionalidade ``média'', com distribuição
bimodal: 50\% das amostras na classe ``baixa'' e 40\% na classe ``alta''
(Tabela 6). Essa heterogeneidade reflete a estratificação vertical
natural dos solos de Cerrado, onde as camadas superficiais (0-20 cm)
concentram a maior parte da matéria orgânica e apresentam densidade
reduzida (média de 0.97 g cm⁻³), enquanto as camadas subsuperficiais
(30-50 cm) apresentam empobrecimento químico e adensamento natural. A
amplitude de valores do FSNSI no Cerrado (1.56 a 8.57, range = 7.01) foi
a maior entre todos os sistemas.istemas, evidenciando gradientes
verticais acentuados.

\textbf{Tabela 6.} Estatísticas descritivas do Índice Fuzzy de
Sustentabilidade Edáfica (FSNSI) por uso da terra, incluindo medidas de
tendência central, dispersão e distribuição das classes linguísticas de
funcionalidade.

\begin{longtable}[]{@{}
  >{\raggedright\arraybackslash}p{(\linewidth - 24\tabcolsep) * \real{0.1120}}
  >{\raggedleft\arraybackslash}p{(\linewidth - 24\tabcolsep) * \real{0.0240}}
  >{\raggedleft\arraybackslash}p{(\linewidth - 24\tabcolsep) * \real{0.1040}}
  >{\raggedleft\arraybackslash}p{(\linewidth - 24\tabcolsep) * \real{0.0400}}
  >{\raggedleft\arraybackslash}p{(\linewidth - 24\tabcolsep) * \real{0.0400}}
  >{\raggedleft\arraybackslash}p{(\linewidth - 24\tabcolsep) * \real{0.0640}}
  >{\raggedleft\arraybackslash}p{(\linewidth - 24\tabcolsep) * \real{0.0480}}
  >{\raggedleft\arraybackslash}p{(\linewidth - 24\tabcolsep) * \real{0.0400}}
  >{\raggedleft\arraybackslash}p{(\linewidth - 24\tabcolsep) * \real{0.0720}}
  >{\raggedleft\arraybackslash}p{(\linewidth - 24\tabcolsep) * \real{0.0400}}
  >{\raggedleft\arraybackslash}p{(\linewidth - 24\tabcolsep) * \real{0.0480}}
  >{\raggedright\arraybackslash}p{(\linewidth - 24\tabcolsep) * \real{0.1680}}
  >{\raggedright\arraybackslash}p{(\linewidth - 24\tabcolsep) * \real{0.2000}}@{}}
\toprule\noalign{}
\begin{minipage}[b]{\linewidth}\raggedright
Uso da Terra
\end{minipage} & \begin{minipage}[b]{\linewidth}\raggedleft
n
\end{minipage} & \begin{minipage}[b]{\linewidth}\raggedleft
FSNSI Médio
\end{minipage} & \begin{minipage}[b]{\linewidth}\raggedleft
DP
\end{minipage} & \begin{minipage}[b]{\linewidth}\raggedleft
EP
\end{minipage} & \begin{minipage}[b]{\linewidth}\raggedleft
CV (\%)
\end{minipage} & \begin{minipage}[b]{\linewidth}\raggedleft
Mín.
\end{minipage} & \begin{minipage}[b]{\linewidth}\raggedleft
Q1
\end{minipage} & \begin{minipage}[b]{\linewidth}\raggedleft
Mediana
\end{minipage} & \begin{minipage}[b]{\linewidth}\raggedleft
Q3
\end{minipage} & \begin{minipage}[b]{\linewidth}\raggedleft
Máx.
\end{minipage} & \begin{minipage}[b]{\linewidth}\raggedright
Classe Predominante
\end{minipage} & \begin{minipage}[b]{\linewidth}\raggedright
Distribuição de Classes
\end{minipage} \\
\midrule\noalign{}
\endhead
\bottomrule\noalign{}
\endlastfoot
Eucalipto & 40 & 6.07 & 2.78 & 0.44 & 45.8 & 1.63 & 3.98 & 7.46 & 8.35 &
8.57 & Alta & Alta (60\%); Baixa (25\%); Média (15\%) \\
Mogno-africano & 40 & 4.54 & 2.89 & 0.46 & 63.6 & 1.50 & 1.61 & 4.18 &
8.30 & 8.66 & Baixa & Baixa (40\%); Média (30\%); Alta (30\%) \\
Cerrado & 40 & 4.42 & 2.89 & 0.46 & 65.3 & 1.56 & 1.62 & 4.16 & 6.88 &
8.57 & Baixa & Baixa (50\%); Alta (40\%); Média (10\%) \\
Agricultura & 40 & 4.25 & 2.57 & 0.41 & 60.6 & 1.31 & 1.66 & 4.18 & 5.21
& 8.21 & Baixa & Baixa (37.5\%); Média (37.5\%); Alta (22.5\%) \\
Teca & 40 & 2.77 & 1.16 & 0.18 & 41.8 & 1.42 & 1.71 & 2.82 & 3.80 & 4.55
& Baixa & Baixa (62.5\%); Média (32.5\%) \\
\end{longtable}

DP: desvio-padrão; EP: erro-padrão; CV: coeficiente de variação; Q1:
primeiro quartil; Q3: terceiro quartil.

A Agricultura convencional registrou FSNSI médio de 4.25, com
distribuição equilibrada entre classes ``baixa'' e ``média'' (37.5\%
cada), indicando funcionalidade comprometida, mas não colapsada. O
coeficiente de variação moderado (60.6\%) sugere certa homogeneização
vertical do perfil edáfico decorrente do preparo mecanizado recorrente
(aração e gradagem), que promove mistura das camadas e reduz os
gradientes naturais. No entanto, 22.5\% das amostras ainda atingiram
classificação ``alta'', possivelmente associadas às camadas superficiais
recém-adubadas ou a microsítios com acúmulo residual de matéria
orgânica.

O Mogno-africano apresentou desempenho intermediário (FSNSI = 4.54), com
distribuição relativamente equilibrada entre as três classes (40\%
baixa, 30\% média, 30\% alta). Esse padrão sugere transição entre estado
de degradação e recuperação funcional, reflexo do manejo menos intensivo
comparativamente à agricultura. O sistema não diferiu estatisticamente
do Cerrado (p = 0.999) nem da Agricultura (p = 0.986), posicionando-se
como sistema de ``transição ecológica'', com potencial de melhoria da
funcionalidade mediante ajustes no manejo (e.g., manutenção de cobertura
morta, redução de intervenções mecanizadas).

\subsection{3.3.3 Estratificação vertical do
FSNSI}\label{estratificauxe7uxe3o-vertical-do-fsnsi}

A análise estratificada por profundidade revelou gradientes verticais
acentuados no FSNSI, com padrões distintos entre os usos da terra
(Figura 10). De modo geral, observou-se decréscimo significativo da
funcionalidade edáfica com o aumento da profundidade, reflexo da redução
conjunta dos estoques de N e P e do aumento da densidade do solo nas
camadas subsuperficiais. As camadas 0-10 cm e 10-20 cm apresentaram
sistematicamente os maiores valores de FSNSI, concentrando 78\% das
amostras classificadas como funcionalidade ``alta''. No Cerrado nativo,
a dicotomia superfície-subsuperfície foi acentuada: FSNSI médio de 7.04
± 0.05 em 0-10 cm (funcionalidade ``alta'') versus 1.59 ± 0.03 em 40-50
cm (funcionalidade ``baixa''). Esse gradiente vertical de 445\% reflete
a arquitetura biogeoquímica natural dos Latossolos sob Cerrado, onde o
sistema radicular superficial promove ciclagem intensa de nutrientes nas
camadas superiores, enquanto as camadas profundas apresentam elevada
intemperização, lixiviação histórica de nutrientes e adensamento coesivo
natural.

\begin{figure}

{\centering \includegraphics[width=0.95\linewidth,height=\textheight,keepaspectratio]{chapters/../../2-FIGURAS/fsnsi_profundidade_uso_terra.png}

}

\caption{Figura 10. Distribuição do FSNSI estratificado por profundidade
e uso da terra. Padrões visuais distintos identificam cada sistema.
Observa-se decréscimo sistemático da funcionalidade edáfica com o
aumento da profundidade em todos os sistemas avaliados.}

\end{figure}%

O Eucalipto manteve FSNSI elevado até 20 cm de profundidade (média de
8.15 em 0-10 cm e 8.18 em 10-20 cm), mas apresentou declínio abrupto a
partir de 20 cm (3.97 em 20-30 cm), sugerindo concentração superficial
dos benefícios do aporte de serapilheira. A Agricultura exibiu padrão
similar, com valores altos nas camadas superficiais devido à adubação
(8.05 em 0-10 cm), mas colapso funcional a partir de 30 cm (1.68 em
30-40 cm), evidenciando que os benefícios do manejo químico não alcançam
as camadas subsuperficiais.

A Teca apresentou o padrão mais homogêneo verticalmente, com pouca
variação entre profundidades (range = 1.79 a 3.35), sugerindo que a
degradação funcional é generalizada ao longo de todo o perfil. Esse
resultado reforça a necessidade de intervenções de manejo específicas
para essa espécie, incluindo adubação de manutenção, incorporação de
matéria orgânica e práticas de descompactação mecânica ou biológica.

\subsection{3.3.4 Correlações entre variáveis de entrada e
FSNSI}\label{correlauxe7uxf5es-entre-variuxe1veis-de-entrada-e-fsnsi}

A análise de correlação de Pearson (Tabela 7) confirmou que N total e P
total foram os preditores mais fortemente associados ao FSNSI (r =
0.789, p \textless{} 0.001 para ambos), explicando aproximadamente 62\%
da variância do índice (r² = 0.622). Esse resultado valida a estrutura
lógica do sistema fuzzy, demonstrando que a disponibilidade química de
nutrientes constitui o fator primário na determinação da funcionalidade
edáfica. As frações lábeis e húmicas de N e P também apresentaram
correlações positivas significativas (r = 0.616-0.627, p \textless{}
0.001), reforçando a interdependência funcional entre compartimentos
químicos.

\textbf{Tabela 7.} Coeficientes de correlação de Pearson entre variáveis
edáficas e o Índice Fuzzy de Sustentabilidade Edáfica (FSNSI), incluindo
intervalos de confiança 95\% e níveis de significância estatística.

\begin{longtable}[]{@{}
  >{\raggedright\arraybackslash}p{(\linewidth - 8\tabcolsep) * \real{0.1282}}
  >{\raggedleft\arraybackslash}p{(\linewidth - 8\tabcolsep) * \real{0.2051}}
  >{\centering\arraybackslash}p{(\linewidth - 8\tabcolsep) * \real{0.3590}}
  >{\raggedright\arraybackslash}p{(\linewidth - 8\tabcolsep) * \real{0.1154}}
  >{\raggedright\arraybackslash}p{(\linewidth - 8\tabcolsep) * \real{0.1923}}@{}}
\toprule\noalign{}
\begin{minipage}[b]{\linewidth}\raggedright
Variável
\end{minipage} & \begin{minipage}[b]{\linewidth}\raggedleft
Correlação (r)
\end{minipage} & \begin{minipage}[b]{\linewidth}\centering
Intervalo de Confiança 95\%
\end{minipage} & \begin{minipage}[b]{\linewidth}\raggedright
p-valor
\end{minipage} & \begin{minipage}[b]{\linewidth}\raggedright
Significância
\end{minipage} \\
\midrule\noalign{}
\endhead
\bottomrule\noalign{}
\endlastfoot
N total & +0.789 & 0.730 a 0.837 & \textless{} 2e-16 & *** \\
P total & +0.789 & 0.730 a 0.837 & \textless{} 2e-16 & *** \\
N lábil & +0.627 & 0.533 a 0.705 & \textless{} 2e-16 & *** \\
P lábil & +0.627 & 0.533 a 0.705 & \textless{} 2e-16 & *** \\
N húmico & +0.616 & 0.521 a 0.696 & \textless{} 2e-16 & *** \\
P húmico & +0.616 & 0.533 a 0.705 & \textless{} 2e-16 & *** \\
Densidade do solo & -0.424 & -0.545 a -0.318 & 1.15e-10 & *** \\
Areia & -0.439 & -0.545 a -0.318 & \textless{} 2e-16 & *** \\
Argila & +0.267 & 0.094 a 0.359 & 0.001 & ** \\
Silte & +0.230 & 0.132 a 0.392 & \textless{} 0.001 & *** \\
\end{longtable}

*** p \textless{} 0.001; ** p \textless{} 0.01; * p \textless{} 0.05.

A densidade do solo apresentou correlação negativa moderada com o FSNSI
(r = -0.424, p \textless{} 0.001), confirmando seu papel como fator
limitante transversal. Esse resultado é particularmente relevante, pois
demonstra que a degradação física pode comprometer a funcionalidade
edáfica mesmo em condições químicas favoráveis. Os mecanismos pelos
quais a compactação afeta a funcionalidade edáfica incluem: (i) redução
da porosidade total e, particularmente, dos macroporos (\textgreater{}
50 μm), limitando a difusão de O₂ para horizontes subsuperficiais e
criando condições de hipoxia que favorecem processos anaeróbicos de
desnitrificação (N₂O, N₂) e redução de formas oxidadas de P ligadas a
Fe³⁺, com consequente volatilização de N e solubilização excessiva de P
que pode ser perdido por lixiviação; (ii) restrição ao crescimento
radicular, especialmente para espécies com sistemas de raízes
pivotantes, limitando a exploração de camadas subsuperficiais ricas em
nutrientes lixiviados e reduzindo a bioperturbação que naturalmente
descompacta o solo; (iii) diminuição da atividade microbiana aeróbica
devido à limitação de O₂, comprometendo processos-chave como a
mineralização de N orgânico, a solubilização de P por ácidos orgânicos
exsudados por microrganianos e a humificação progressiva de resíduos
vegetais; e (iv) redução da infiltração de água e aumento do escoamento
superficial, favorecendo erosão laminar que remove seletivamente as
frações finas do solo (argila, silte) ricas em MOS e nutrientes
associados.

Estudos sobre mudança de uso da terra e frações húmicas corroboram esses
padrões, demonstrando que a conversão de ecossistemas nativos altera
profundamente o particionamento de N entre compartimentos funcionais.
Santos et al. (2024) evidenciaram em Latossolo Vermelho sob diferentes
coberturas no Tocantins que a Teca apresentou os maiores estoques de N
total (11.47 Mg ha⁻¹), enquanto o Cerrado nativo exibiu os menores
(10.16 Mg ha⁻¹), padrão consistente com os resultados do presente
estudo. Sistemas florestais plantados concentraram formas mais estáveis
(ácidos húmicos e humina), associadas ao maior input de serapilheira
lignificada, enquanto a agricultura favoreceu acúmulo de formas lábeis
(ácidos fúlvicos). A análise de redundância (RDA) explicou 47\% da
variação total, evidenciando gradientes de enriquecimento e
estabilização de N que se alinham com os coeficientes estruturais β
elevados observados no PLS-SEM. Esses achados reforçam que práticas
conservacionistas baseadas em plantios arbóreos com raízes profundas e
input orgânico contínuo são essenciais para preservar a estabilidade de
N e a sustentabilidade de Latossolos tropicais, especialmente em
contextos de intensificação agrícola where soil physical degradation
compromises chemical functionality.

Estudos prévios corroboram esses mecanismos: em Latossolos com densidade
\textgreater{} 1.5 g cm⁻³, a taxa de mineralização de N orgânico
reduz-se em 35-48\% comparativamente a solos bem estruturados (densidade
\textless{} 1.2 g cm⁻³), enquanto a emissão de N₂O aumenta em 2-3 vezes
devido à predominância de microsítios anaeróbicos. A compactação também
altera a distribuição espacial das frações de P: em solos compactados,
observa-se acúmulo relativo de P lábil nas camadas superficiais (0-10
cm) devido à redução da infiltração, enquanto as frações húmicas de P
concentram-se em profundidade, criando gradientes verticais acentuados
que comprometem a eficiência de uso do nutriente pelas plantas. Essa
estratificação vertical excessiva é capturada pelo FSNSI através da
penalização imposta pela densidade, que opera multiplicativamente sobre
os índices de disponibilidade química: solos com densidade de 1.6 g cm⁻³
(observada em Teca e Pastagem) recebem penalização de aproximadamente
40\% no FSNSI, mesmo apresentando teores adequados de N e P totais.

A textura do solo também influenciou o FSNSI, com areia
correlacionando-se negativamente (r = -0.439, p \textless{} 0.001) e
argila positivamente (r = 0.267, p = 0.001), reflexo da maior capacidade
de retenção de nutrientes e água em solos argilosos. Esse padrão é
consistente com a teoria de proteção físico-química da MOS, segundo a
qual frações finas do solo (argila \textless{} 2 μm, silte 2-50 μm)
formam complexos organo-minerais estáveis que protegem N e P contra
mineralização rápida. Em solos arenosos, a baixa superfície específica
(\textless{} 5 m² g⁻¹) e a ausência de minerais reativos (predominância
de quartzo inerte) limitam a adsorção de substâncias húmicas e a
formação de microagregados, resultando em MOS menos estável e maior
suscetibilidade à mineralização. A correlação positiva entre argila e
FSNSI (r = 0.267) demonstra que, mesmo em Latossolos altamente
intemperizados onde a mineralogia é dominada por caulinita de baixa
atividade, o incremento no teor de argila ainda confere benefícios
funcionais mensuráveis, provavelmente mediados pela maior retenção de
água e pelo aumento da superfície disponível para adsorção de nutrientes
e colonização microbiana.

Esses resultados reforçam a adequação da abordagem fuzzy para integração
de múltiplas dimensões da funcionalidade edáfica, capturando
simultaneamente aspectos químicos (disponibilidade de nutrientes),
físicos (compactação) e mineralógicos (textura). A lógica fuzzy permitiu
traduzir essas relações quantitativas em um índice interpretável,
facilitando a identificação de sistemas prioritários para intervenções
de manejo conservacionista e recuperação da qualidade do solo. A
validação dessas relações por meio de análise de correlação multivariada
demonstra que o FSNSI não constitui mera síntese empírica, mas reflete
processos biogeoquímicos fundamentais que governam a funcionalidade
edáfica em ambientes tropicais. A magnitude das correlações (r =
0.62-0.79 para variáveis químicas, r = -0.42 para densidade) situa-se na
faixa esperada para sistemas pedológicos complexos onde múltiplos
fatores interagem de forma não linear, validando a robustez do sistema
fuzzy como ferramenta diagnóstica integrativa.

A combinação de PLSR, PLS-SEM e inferência fuzzy sintetizou a dinâmica
de N e P em múltiplas escalas, alinhando-se às recomendações
metodológicas para sistemas complexos (Hair et al., 2021). A etapa PLSR
destacou as frações húmicas como principais preditoras dos estoques
totais, enquanto o PLS-SEM quantificou a dominância dos compartimentos
NTAH, NTHum, PTAH e PTHum (β = 1.286 para N\_húmico → N\_total), em
consonância com evidências de que substâncias húmicas constituem os
principais reservatórios estáveis em Latossolos manejados (Pegoraro et
al., 2011, Ferreira et al., 2021). Em contraste, coeficientes negativos
atribuídos às frações lábeis refletem a transferência contínua desses
pools para formas mais recalcitrantes, dinâmica também reportada em
cronossequências tropicais (Marinho Junior et al., 2021).

O FSNSI expôs diferenças funcionais claras entre os usos da terra em
que, o Eucalipto apresentou índice superior ao Cerrado (6.07 versus
4.42), compatível com a capacidade dessa espécie de enriquecer frações
húmicas e melhorar a estrutura física quando resíduos não são removidos
(Crespo et al., 2024). A Teca exibiu o menor desempenho (2.77),
reforçando que serapilheiras altamente recalcitrantes e compactação
residual limitam a reciclagem de nutrientes e elevam a penalização
física do índice (Santos et al., 2024). Sistemas agrícolas mantiveram
valores intermediários (4.25), indicando que a adubação mineral sustenta
a disponibilidade superficial, mas não reverte a degradação das camadas
subsuperficiais.

As correlações confirmaram que N total e P total explicam parte
substancial da variação do FSNSI (r = 0.789), porém a densidade do solo
relacionou-se negativamente com o índice (r = -0.424), ressaltando que a
degradação física continua sendo limitante transversal mesmo em solos
quimicamente enriquecidos (Santos et al., 2024). Assim, a interpretação
integrada das três abordagens evidencia que indicadores isolados podem
mascarar desequilíbrios, enquanto o FSNSI agrega dimensões químicas e
físicas coerentes com a arquitetura estrutural revelada pelo PLS-SEM,
oferecendo ferramenta robusta para priorização de ações de manejo e
restauração.

\section*{Articulação entre PLSR, PLS-SEM e inferência
fuzzy}\label{articulauxe7uxe3o-entre-plsr-pls-sem-e-inferuxeancia-fuzzy}
\addcontentsline{toc}{section}{Articulação entre PLSR, PLS-SEM e
inferência fuzzy}

\markright{Articulação entre PLSR, PLS-SEM e inferência fuzzy}

A integração das três abordagens analíticas (PLSR, PLS-SEM e inferência
fuzzy) permitiu compreensão multiescalar e multidimensional da dinâmica
de N e P nos solos estudados. A análise PLSR identificou as frações mais
relevantes para predição dos estoques totais, fornecendo subsídios para
seleção de indicadores de monitoramento. O modelo PLS-SEM explicitou as
relações causais entre compartimentos funcionais, quantificando as
contribuições relativas das formas lábeis e húmicas. Por fim, o sistema
fuzzy traduziu essas informações em índice sintético de funcionalidade,
incorporando a dimensão física (densidade do solo) como fator limitante.

A estrutura do modelo PLS-SEM demonstra que as frações húmicas (NTAH,
NTHum, PTAH, PTHum) funcionam como reservatórios preferenciais para N e
P, com contribuições dominantes na predição dos pools totais. Os
coeficientes de caminho (β = 1.286 para N\_húmico → N\_total, β = -0.313
para N\_lábil → N\_total no ambiente Cerrado) evidenciam interações
entre compartimentos, onde o enriquecimento da fração húmica amplifica o
efeito no estoque total, enquanto a fração lábil apresenta relação
negativa, refletindo dinâmica de particionamento fisiológico entre
pools. A magnitude variável desses coeficientes entre os cinco ambientes
(Cerrado, Agricultura, Mogno, Eucalipto, Teca) reforça a hipótese de que
o manejo e a cobertura vegetal modulam as trajetórias de ciclagem de
nutrientes.

A análise fuzzy revelou padrão contrastante: o Eucalipto apresentou
funcionalidade edáfica superior ao Cerrado nativo (FSNSI de 6.07 versus
4.42, p = 0.036), resultado que pode ser explicado pela natureza não
manejada do povoamento de Eucalipto avaliado (11 anos sem colheita), que
permitiu acúmulo contínuo de serapilheira rica em compostos
recalcitrantes, promovendo humificação intensa e enriquecimento das
frações húmicas de N e P. Adicionalmente, o sistema radicular do
Eucalipto promove bioperturbação que reduz localmente a densidade
aparente, atenuando a penalização física no cálculo do FSNSI.

Essa interpretação articula-se com os resultados das análises PLSR e
PLS-SEM: se as frações húmicas são preditoras-chave dos estoques totais
(VIP \textgreater{} 1.0) e o Eucalipto favorece seu acúmulo, então é
coerente que esse sistema expresse funcionalidade edáfica elevada. A
aparente contradição com a expectativa de superioridade do Cerrado
reside no fato de que o FSNSI captura funcionalidade bioquímica
corrente, não necessariamente equilíbrio ecológico de longo prazo ou
biodiversidade edáfica. Sistemas florestais plantados podem, sob
condições específicas (ausência de colheita, baixa intensidade de
manejo), promover acúmulo transitório de nutrientes superior ao de
ecossistemas naturais oligotróficos como o Cerrado.

O desempenho da Teca (FSNSI = 2.77, 54\% inferior ao Eucalipto, p
\textless{} 0.001) evidencia o papel determinante da qualidade da
serapilheira e do manejo pós-implantação. A Teca, espécie decídua com
folhas de relação C:N elevada e baixa taxa de decomposição, promove
aporte de resíduos de qualidade inferior para humificação.
Adicionalmente, a compactação residual do preparo mecanizado (densidade
média de 1.62 g cm⁻³) não foi mitigada ao longo do tempo, resultando em
penalização física persistente no FSNSI. Esse resultado indica
necessidade de práticas complementares em sistemas de Teca, incluindo
consórcios com leguminosas fixadoras de N, adubação orgânica e
descompactação biológica.

A Agricultura convencional posicionou-se em estado intermediário (FSNSI
= 4.25), não diferindo estatisticamente do Cerrado (p = 0.998) nem do
Mogno (p = 0.986). Esse resultado sugere que, sob manejo químico
adequado (adubação NPK), é possível manter funcionalidade edáfica
comparável à vegetação nativa, ao menos em termos de disponibilidade
química de nutrientes. No entanto, a distribuição bimodal das classes
(37.5\% ``baixa'' e 37.5\% ``média'') revela heterogeneidade vertical
pronunciada, com funcionalidade concentrada nas camadas superficiais
adubadas e colapso nas camadas subsuperficiais, padrão não observado no
Cerrado. Esse contraste reforça que a adubação mineral compensa
parcialmente a degradação química, mas não restaura os gradientes
verticais naturais nem a estrutura física do solo.

A análise de correlação confirmou que N total e P total explicam 62\% da
variância do FSNSI (r = 0.789, r² = 0.622), validando a estrutura lógica
do sistema fuzzy. No entanto, a correlação negativa da densidade do solo
(r = -0.424, p \textless{} 0.001) demonstra que a degradação física atua
como fator limitante transversal, reduzindo o FSNSI mesmo em solos
quimicamente enriquecidos. Esse resultado tem implicação prática:
estratégias de manejo devem considerar simultaneamente fertilidade
química e qualidade física, pois o comprometimento de qualquer dimensão
restringe a funcionalidade global do sistema.

Essa síntese integrativa reforça a necessidade de abordagens
multiescalares para avaliação da sustentabilidade edáfica em ambientes
tropicais. Indicadores isolados podem mascarar desequilíbrios funcionais
que somente emergem quando se considera simultaneamente a
compartimentalização química, as relações estruturais entre pools de
nutrientes quantificadas por modelagem PLS-SEM, e a integração com
atributos físicos restritivos (densidade do solo como penalizador). O
FSNSI, por incorporar essas múltiplas dimensões em índice interpretável
e por ter sua fundamentação validada pela correspondência com os
caminhos de análise PLS-SEM, constitui ferramenta adequada para
monitoramento da funcionalidade edáfica e identificação de sistemas
prioritários para restauração ecológica.

\bookmarksetup{startatroot}

\chapter{}\label{section}

\# Conclusão

A integração metodológica (PLSR, PLS-SEM e inferência fuzzy) permitiu a
caracterização estrutural e funcional da dinâmica de N e P em solos
tropicais, revelando a heterogeneidade induzida pelo manejo.

As frações húmicas atuam como preditoras-chave da disponibilidade de
nutrientes, funcionando como integradoras de múltiplos processos
biogeoquímicos e sugerindo um acoplamento geoquímico fundamental em
Latossolos.

A resposta funcional do solo é transversalmente limitada por fatores
físicos, como a densidade, estabelecendo que a degradação física não é
compensada por um elevado desempenho químico.

A divergência de desempenho entre sistemas (e.g., Eucalipto superando
Cerrado nativo) questiona paradigmas conservacionistas, indicando que a
funcionalidade é determinada pela interação dinâmica entre aporte de
matéria orgânica, práticas de manejo e legado pedológico.

A funcionalidade bioquímica elevada não garante a sustentabilidade em
regimes de exploração intensiva, como evidenciado pela redução funcional
em sistemas com Teca.

O FSNSI (Índice de Sustentabilidade Funcional do Solo e Nutrientes)
demonstrou ser uma ferramenta diagnóstica multiescalar, mas requer
adaptação pedoclimática local e integração com indicadores biológicos
para um diagnóstico completo de sustentabilidade.

Recomendações de manejo devem ser ajustadas ao arranjo de uso e
acompanhadas por monitoramento em profundidade para antecipar colapsos
subsuperficiais. Estudos futuros devem focar nas trajetórias temporais
para identificar transições funcionais críticas.

\bookmarksetup{startatroot}

\chapter*{Referências}\label{referuxeancias}
\addcontentsline{toc}{chapter}{Referências}

\markboth{Referências}{Referências}

\phantomsection\label{refs}
\begin{CSLReferences}{1}{0}
\bibitem[\citeproctext]{ref-almeida2024}
Almeida, D. C. et al. (2024). Mudança de uso da terra e seus efeitos nos
estoques de carbono e nitrogênio do solo em Latossolo do Cerrado
brasileiro. \emph{Revista Brasileira de Ciência do Solo}, \textbf{48} p.
e0240128.

\bibitem[\citeproctext]{ref-araujofilho2024}
Araújo Filho, R. N. de, M. B. G. dos S. Freire, F. J. Freire, R. L. C.
Ferreira, L. M. Pereira \& L. D. V. Santos (2024). Litter Dynamics and
Nutrient Stocks in a Chronosequence of Hyperxerophilous Forest under the
Effect of Clear Felling. \emph{Forest Ecology and Management},
\textbf{567} p. 122087.
\url{https://doi.org/10.1016/j.foreco.2024.122087}

\bibitem[\citeproctext]{ref-camposmontanari2024}
Campos, A. \& B. Montanari (2024). Limpeza de área em plantios
florestais. \emph{Revista de Silvicultura}.

\bibitem[\citeproctext]{ref-chen2018}
Chen, R., M. Senbayram, S. Blagodatsky, O. Myachina, K. Dittert, C. Liu,
et al. (2018). Soil C and N availability determine the priming effect:
Microbial N mining and stoichiometric decomposition theories.
\emph{Global Change Biology}, \textbf{24}, (12) p. 5122--5134.
\url{https://doi.org/10.1111/gcb.14474}

\bibitem[\citeproctext]{ref-cho2021}
Cho, H. S. et al. (2021). Topographic analysis of soil properties.
\emph{Soil Science Society of America Journal}, \textbf{85} p.

\bibitem[\citeproctext]{ref-cotrufo2019}
Cotrufo, M. F., M. G. Ranalli, M. L. Haddix, J. Six \& E. Lugato (2019).
Soil carbon storage informed by particulate and mineral-associated
organic matter. \emph{Nature Geoscience}, \textbf{12} p. 989--994.
\url{https://doi.org/10.1038/s41561-019-0484-6}

\bibitem[\citeproctext]{ref-crespo2024}
Crespo, C. M. G., V. C. Piscoya, R. C. P. de Melo, L. M. Pereira, L. D.
V. Santos, F. S. R. Holanda, et al. (2024). Topographic Modulation of
Soil Functional Indicators in Shaded Coffee Agroforestry Systems: A
Multivariate and Network-Based Approach. \emph{Agronomy}, \textbf{14} p.
2847. \url{https://doi.org/10.3390/agronomy14122847}

\bibitem[\citeproctext]{ref-eckardt2021}
Eckardt, J. et al. (2021). Cultivo de milho no cerrado. \emph{Journal of
Agriculture}.

\bibitem[\citeproctext]{ref-ferreira2021}
Ferreira, G. W. D., F. C. C. Oliveira, E. M. B. Soares, J. Schnecker, I.
R. Silva \& A. S. Grandy (2021). Retaining eucalyptus harvest residues
promotes different pathways for particulate and mineral‐associated
organic matter. \emph{Ecosphere}, \textbf{12}.
\url{https://doi.org/10.1002/ecs2.3439}

\bibitem[\citeproctext]{ref-hair2021}
Hair, J. F., G. T. M. Hult, C. M. Ringle \& M. Sarstedt (2021). \emph{A
primer on Partial Least Squares Structural Equation Modeling (PLS-SEM)}.
Sage Publications: 3º ed.

\bibitem[\citeproctext]{ref-helfenstein2020}
Helfenstein, J., J. Jegminat, T. I. McLaren, E. Frossard \& A. Oberson
(2020). Soil organic phosphorus transformations in response to
management and environmental changes: A review. \emph{Soil Biology and
Biochemistry}, \textbf{156} p. 108193.
\url{https://doi.org/10.1016/j.soilbio.2021.108193}

\bibitem[\citeproctext]{ref-ker1997}
Ker, J. C. (1997). Latossolos do Brasil: uma revisão. \emph{Geonomos},
\textbf{5}, (1) p. 17--40.

\bibitem[\citeproctext]{ref-klink2005}
Klink, C. A. \& R. B. Machado (2005). Conservation of the Brazilian
Cerrado. \emph{Conservation Biology}, \textbf{19}, (3) p. 707--713.

\bibitem[\citeproctext]{ref-lal2020}
Lal, R. (2020). Soil organic matter and water retention: The impact of
land use and management. \emph{Land Degradation and Development},
\textbf{31}, (9) p. 1369--1382. \url{https://doi.org/10.1002/ldr.3580}

\bibitem[\citeproctext]{ref-leal2024}
Leal, O. dos A., G. S. Santana, H. Knicker, F. J. González-Vila, J. A.
González-Pérez \& D. P. Dick (2024). Acacia and Eucalyptus plantations
modify the molecular composition of density organic matter fractions of
subtropical native pasture soils. \emph{Geoderma}, \textbf{441} p.
116745. \url{https://doi.org/10.1016/j.geoderma.2023.116745}

\bibitem[\citeproctext]{ref-lehmann2015}
Lehmann, J. \& M. Kleber (2015). The contentious nature of soil organic
matter. \emph{Nature}, \textbf{528} p. 60--68.
\url{https://doi.org/10.1038/nature16069}

\bibitem[\citeproctext]{ref-lima2023}
Lima, J. R. S., F. V. Andrade, M. F. P. Santos \& E. S. Mendonça (2023).
Integrating fuzzy logic and soil indicators to assess the sustainability
of tropical agroecosystems. \emph{Ecological Indicators}, \textbf{150}
p. 110234. \url{https://doi.org/10.1016/j.ecolind.2023.110234}

\bibitem[\citeproctext]{ref-lopes1996}
Lopes, A. S. \& L. R. G. Guilherme (1996). A career perspective on soil
management in the Cerrado region of Brazil. \emph{Advances in Agronomy},
\textbf{57} p. 1--64.

\bibitem[\citeproctext]{ref-machado2024}
Machado, I. E. S. et al. (2024). Sistemas de culturas para o cultivo de
soja no cerrado. \emph{Revista Caderno Pedagógico}, \textbf{21}, (7) p.
1--21.

\bibitem[\citeproctext]{ref-mamdani1977}
Mamdani, E. H. (1977). Application of fuzzy logic to approximate
reasoning using linguistic synthesis. \emph{IEEE Transactions on
Computers}, \textbf{26}, (12) p. 1182--1191.
\url{https://doi.org/10.1109/TC.1977.1674779}

\bibitem[\citeproctext]{ref-marinhojunior2021}
Marinho Junior, J. L., V. C. Piscoya, M. M. Fernandes, S. B. Gonçalves,
F. S. R. Holanda, M. Cunha Filho, et al. (2021). Carbon Dynamics in
Humic Fractions of Soil Organic Matter Under Different Vegetation Cover
in Southern Tocantins. \emph{Floresta e Ambiente}, \textbf{28}.
\url{https://doi.org/10.1590/2179-8087-floram-2020-0024}

\bibitem[\citeproctext]{ref-mendonca2024}
Mendonça, E. S., J. R. S. Lima \& A. L. Ferreira (2024). Fuzzy-based
modeling of soil multifunctionality under land use intensification in
tropical regions. \emph{Environmental Modelling and Software},
\textbf{173} p. 105783.
\url{https://doi.org/10.1016/j.envsoft.2024.105783}

\bibitem[\citeproctext]{ref-moreira2021}
Moreira, M. F. et al. (2021). Teca: implantação e produção no Brasil.
\emph{Arrudea}, \textbf{7} p. 73--82.

\bibitem[\citeproctext]{ref-paul2016}
Paul, E. A. (2016). The nature and dynamics of soil organic matter:
Reconsiderations within the current paradigm. \emph{Soil Biology and
Biochemistry}, \textbf{98} p. 109--123.
\url{https://doi.org/10.1016/j.soilbio.2016.04.001}

\bibitem[\citeproctext]{ref-pegoraro2011}
Pegoraro, R. F., I. R. da Silva, R. F. de Novais, N. F. de Barros, S.
Fonseca \& C. S. Dambroz (2011). Estoques de carbono e nitrogênio nas
frações da matéria orgânica em argissolo sob eucalipto e pastagem.
\emph{Ciência Florestal}, \textbf{21} p. 261--273.
\url{https://doi.org/10.5902/198050983230}

\bibitem[\citeproctext]{ref-queiroz2017}
Queiroz, W. T. et al. (2017). Índice de Valor de Importância de Espécies
Arbóreas da Floresta Nacional do Tapajós Via Análises de Componentes
Principais e de Fatores. \emph{Ciência Florestal}, \textbf{27}, (1) p.
47--59.

\bibitem[\citeproctext]{ref-sano2019}
Sano, E. E. et al. (2019). Cerrado ecoregions: A spatial framework to
assess and prioritize Brazilian savanna environmental diversity for
conservation. \emph{Journal of Environmental Management}, \textbf{232}
p. 818--828.

\bibitem[\citeproctext]{ref-sano2020}
Sano, E. E., A. A. Rodrigues, E. S. Martins \& G. M. Bettiol (2020).
Cerrado e suas transformações: Dinâmica da ocupação e desafios à
conservação. \emph{Revista Brasileira de Geografia Física}, \textbf{13},
(2) p. 442--459. \url{https://doi.org/10.26848/rbgf.v13.2.p442-459}

\bibitem[\citeproctext]{ref-santos2024}
Santos, L. D. V., L. M. Pereira, S. A. B. da Silva, F. S. R. Holanda, R.
C. P. de Melo, M. Cunha Filho, et al. (2024). Land Use Change and Its
Effects on Soil Nitrogen Stocks and Humic Fractions in Latosol from
Brazilian Cerrado. \emph{Soil Science Society of America Journal},
\textbf{88} p. 1234--1248. \url{https://doi.org/10.1002/saj2.20567}

\bibitem[\citeproctext]{ref-sekaran2020}
Sekaran, U., J. R. Loya, G. O. Abagandura, S. Subramanian, V. Owens \&
S. Kumar (2020). Intercropping of kura clover (Trifolium ambiguum M.
Bieb) with prairie cordgrass (Spartina pectinata link.) enhanced soil
biochemical activities and microbial community structure. \emph{Applied
Soil Ecology}, \textbf{147}.
\url{https://doi.org/10.1016/j.apsoil.2019.103427}

\bibitem[\citeproctext]{ref-silva2024}
Silva, B. de O., M. R. Moitinho, A. R. Panosso, D. M. da S. Oliveira, R.
Montanari, M. L. T. de Moraes, et al. (2024). Implications of converting
native forest areas to agricultural systems on the dynamics of CO2
emission and carbon stock in a Cerrado soil, Brazil. \emph{Journal of
Environmental Management}, \textbf{358} p. 120796.
\url{https://doi.org/10.1016/j.jenvman.2024.120796}

\bibitem[\citeproctext]{ref-silva2022}
Silva, R. F., A. M. Siqueira \& A. P. Andrade (2022). Land-use
intensification reduces soil organic matter stability in tropical
regions. \emph{Science of the Total Environment}, \textbf{821} p.
153289. \url{https://doi.org/10.1016/j.scitotenv.2022.153289}

\bibitem[\citeproctext]{ref-silva2023}
Silva, R. R. \& S. Barreira (2023). DESENVOLVIMENTO DE Khaya
grandifoliola C. Dc.SOB DIFERENTES ESPAÇAMENTOS DE PLANTIO.
\emph{AGRARIAN ACADEMY}, \textbf{10}, (19) p. 91.

\bibitem[\citeproctext]{ref-sisti2004}
Sisti, C. P. J. et al. (2004). Change in carbon and nitrogen stocks in
soil under 13 years of conventional or zero tillage in southern Brazil.
\emph{Soil and Tillage Research}, \textbf{76}, (1) p. 39--58.

\bibitem[\citeproctext]{ref-sousa2021}
Sousa, D. M. G., H. G. Santos \& J. S. Carneiro (2021). Long-term land
use change impacts on nutrient cycling and soil organic matter fractions
in the Cerrado. \emph{Agriculture, Ecosystems and Environment},
\textbf{319} p. 107567. \url{https://doi.org/10.1016/j.agee.2021.107567}

\bibitem[\citeproctext]{ref-stevenson1994}
Stevenson, F. J. (1994). \emph{Humus chemistry: Genesis, composition,
reactions}. Wiley: 2º ed.

\bibitem[\citeproctext]{ref-strassburg2017}
Strassburg, B. B. N., T. Brooks, R. Feltran-Barbieri, et al. (2017).
Moment of truth for the Cerrado hotspot. \emph{Nature Ecology and
Evolution}, \textbf{1} p. 0099.
\url{https://doi.org/10.1038/s41559-017-0099}

\bibitem[\citeproctext]{ref-teixeira2017}
Teixeira, P. C. et al. (2017). \emph{Manual e métodos de análise de
solo}. Brasília, DF: Embrapa: 3º ed.

\bibitem[\citeproctext]{ref-thornthwaite1948}
Thornthwaite, C. W. (1948). An approach toward a rational classification
of climate. \emph{Geographical Review}, \textbf{38}, (1) p. 55--94.

\bibitem[\citeproctext]{ref-tivet2013}
Tivet, F., J. C. M. Sá, R. Lal, C. Briedis, P. R. Borszowskei \& D. C.
Hartman (2013). Aggregate C depletion by plowing and its restoration by
no-till cropping systems under subtropical and tropical conditions.
\emph{Soil and Tillage Research}, \textbf{126} p. 203--218.
\url{https://doi.org/10.1016/j.still.2012.09.006}

\bibitem[\citeproctext]{ref-verai2022}
Verai, D. E. et al. (2022). Crescimento e forma do eucalipto em função
da densidade de plantio. \emph{Ciência Florestal}, \textbf{32}, (1) p.
504--522.

\bibitem[\citeproctext]{ref-wang2023}
Wang, J., X. Zhang \& X. Wu (2023). Soil phosphorus dynamics under
contrasting land uses in tropical ecosystems: A global synthesis.
\emph{Global Change Biology}, \textbf{29}, (4) p. 1154--1170.
\url{https://doi.org/10.1111/gcb.16435}

\end{CSLReferences}

\begin{center}\rule{0.5\linewidth}{0.5pt}\end{center}

\section*{Como Adicionar
Referências}\label{como-adicionar-referuxeancias}
\addcontentsline{toc}{section}{Como Adicionar Referências}

\markright{Como Adicionar Referências}

Para adicionar mais referências ao artigo, edite o arquivo
\texttt{references.bib} e adicione entradas no formato BibTeX.

\textbf{Exemplo}:

\begin{Shaded}
\begin{Highlighting}[]
\VariableTok{@article}\NormalTok{\{}\OtherTok{autor2020}\NormalTok{,}
  \DataTypeTok{title}\NormalTok{ = \{Título do Artigo\},}
  \DataTypeTok{author}\NormalTok{ = \{Autor, Nome and Coautor, Outro\},}
  \DataTypeTok{journal}\NormalTok{ = \{Nome do Periódico\},}
  \DataTypeTok{year}\NormalTok{ = \{2020\},}
  \DataTypeTok{volume}\NormalTok{ = \{42\},}
  \DataTypeTok{number}\NormalTok{ = \{3\},}
  \DataTypeTok{pages}\NormalTok{ = \{123{-}{-}145\},}
  \DataTypeTok{doi}\NormalTok{ = \{10.1234/exemplo\}}
\NormalTok{\}}
\end{Highlighting}
\end{Shaded}

Então cite no texto usando \texttt{{[}@autor2020{]}}.

\section*{Gestão de Referências}\label{gestuxe3o-de-referuxeancias}
\addcontentsline{toc}{section}{Gestão de Referências}

\markright{Gestão de Referências}

Recomenda-se usar ferramentas como:

\begin{itemize}
\tightlist
\item
  \textbf{Zotero} (gratuito): Exportar coleções em BibTeX
\item
  \textbf{Mendeley}: Integração com R via \texttt{scholar} package
\item
  \textbf{JabRef}: Editor dedicado a arquivos BibTeX
\end{itemize}




\end{document}
